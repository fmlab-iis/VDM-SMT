This section gives studies that apply our encoding to selected VDM-SL models and discusses the results. All the selected VDM-SL models can be found in Overture repository\footnote{http://overturetool.org/download/examples/VDMSL/}.

\subsection{Abstract Pacemaker}

The first case is a small VDM-SL model specifies an abstract pacemaker. The model is not a full specification and focuses on part of definitions and operations related to heartbeat signals detected and recorded in a pacemaker. For this model of 58 lines, Overture generated nine POs. Here we report two the encoding and solving of obligations PO8 and PO3 for this case.

\subsubsection{Proof Obligation 8}

PO8 shown in Fig.~\ref{fig:po8_case1} is a legal sequence application obligation of operation {\tt Pace}.  PO8 requires that the formulas specified with trace {\tt tr} of type {\tt Trace} be legal, i.e., computable to get its result. For more understanding of PO8, we may take a look on the definition of the operation {\tt Pace} as shown in Fig.~\ref{fig:pace_case1}. {\tt Pace} is an explicitly specified operation and takes three inputs of types {\tt Trace}, {\tt nat1}, and {\tt nat1}. It is easy to recognize that PO8 is checking about the sequence used in line 4 in Fig.~\ref{fig:pace_case1}. Note that to discharge PO8, we do not need the full definition of {\tt Pace}.

\begin{figure}[t]
\begin{center}
\begin{mdframed}[roundcorner=5pt]
\begin{Verbatim}[fontsize=\small]
(forall tr:Trace, aperi:nat1, vdel:nat1,
  oldstate & 
    (forall i in set (inds (tl tr)) &
      (((i mod aperi) = (vdel + 1)) =>
        (i in set (inds tr)))
))
\end{Verbatim}
\end{mdframed}
\vspace{-10pt}
\caption{Proof Obligation 8 of Abstract Pacemaker}
\label{fig:po8_case1}
\end{center}
\end{figure}

\begin{figure}[t]
\begin{center}
\begin{mdframed}[roundcorner=5pt]
\begin{Verbatim}[fontsize=\small]
Pace: Trace * nat1 * nat1 ==> Trace
Pace(tr,aperi,vdel) ==
  return [nil] ^
     [ if (i mod aperi = vdel + 1) and
          tr(i) <> <V> 
       then <V> else nil
     | i in set inds tl tr];
\end{Verbatim}
\end{mdframed}
\vspace{-10pt}
\caption{Pace of Abstract Pacemaker}
\label{fig:pace_case1}
\end{center}
\end{figure}

As mentioned in Section~\ref{section:proof-obligations}, PO8 is the formula to be proved and we need to include the context information related to the obligation as the first step of encoding described in Section~\ref{section:encode-strategy}. To decide what to include as the context information, we firstly examine the types used in PO8 and we have {\tt nat1} and {\tt Pace}. {\tt nat1} is a type of natural numbers that excludes zero and thus starts from one. {\tt nat1} is a primitive type of VDM so we do not need more information about it. {\tt Pace} is a user-defined type so we have check its definition in the VDM-SL model. The type definition of {\tt Trace} in the model is as follows. Note that {\tt oldstate} bounded by the forall quantifier is used in the expression of PO8 and therefore is ignored in the encoding.

\begin{mdframed}[roundcorner=5pt]
\begin{Verbatim}[fontsize=\small]
  Trace = seq of [Event];
  Event = <A> | <V>;
\end{Verbatim}
\end{mdframed}

Now we know that {\tt Trace} is of sequence type of {\tt Event}, where {\tt Event} is of quote type with two values {\tt <A>} and {\tt <V>} that are stand for the A-pace and V-pace of heartbeat signals. At this point, we have necessary context information we need to discharge PO8 and we may proceed to encoding step. For discharging PO3, we need to encode types used in PO8: {\tt nat1} and {\tt Trace}. {\tt nat1} is the natural number that excludes zero, so the encoding of {\tt nat1} is as follows. 

\begin{mdframed}[roundcorner=5pt,shadow=true]
\begin{Verbatim}[fontsize=\small]
  aperi = Int('aperi')
  s.add(aperi>=1)
  vdel = Int('vdel')
  s.add(vdel>=1)
\end{Verbatim}
\end{mdframed}

As the above code shows, we defined two instances of integer in Z3, {\tt aperi} and {\tt vdel}, then the constraint of greater and equal to one is added into the solver object for both instances. Note that in Z3, we cannot add constraints to types but instances of a Z3 type. This has been mentioned in Section~\ref{section:encode-strategy}. The remaining type not encoded yet is {\tt Trace}, the sequence type. We defined {\tt Event} firstly since {\tt Trace} is a trace of {\tt Event}. The code below shows the encoding of {\tt Event} and {\tt Trace}. Since {\tt Trace} is not exactly the sequence of {\tt Event} but {\tt [Event]} (a union with {\tt nil}, we did not define {\tt Event} but define {\tt [Event]}. Also, because VDM uses partial functions, a type can be undefined. This means that besides ordinary values as specification shows, we have to consider the case that the value is not defined. In our work, we used {\tt NDF} to represent the undefined value of {\tt Event}, and define {\tt Event\_lift} in Z3. The {\tt \_lift} indicates that we have considered the undefined value.

\begin{mdframed}[roundcorner=5pt,shadow=true]
\begin{Verbatim}[fontsize=\small]
  Event_lift, (A, V, nil, NDF) = 
    EnumSort('Event_lift', 
            ['A', 'V', 'nil', 'NDF'])

  Trace = ArraySort(IntSort(),Event_lift)
\end{Verbatim}
\end{mdframed}

In the above encoding, {\tt Trace} is encoded as an array type in Z3 and this array type is by default indexed through the integer domain. An array type can not be a sequence type unless we add constraints regarding to sequences to limit an array to a sequence. The idea of adding constraints is the same as the encoding of {\tt nat1} but more complicate. Fig.~\ref{fig:tr_constraint_case1} shows the encoding of the constraint to {\tt tr}, the instance of {\tt Trace} as an array. Note again that the constraints are applied on {\tt tr}, the instance of {\tt Trace}, not the type {\tt Trace}. In Fig.~\ref{fig:tr_constraint_case1}, we added three constraints: 

\begin{figure}[t]
\begin{center}
\begin{mdframed}[roundcorner=5pt,shadow=true]
\begin{Verbatim}[fontsize=\small,numbers=left]
  tr = Const('tr',Trace)
  [i,j] = Ints('i j')

  ForAll(i, Implies( i<=0, tr[i]==NDF ) )

  ForAll(i,
    Implies(
      And(i>=1,tr[i]!=NDF),
      ForAll(j,
        Implies(And(j>=1,j<=i),tr[j]!=NDF)
      )
    )
  )

  ForAll(i,
    Implies(
      And(i>=1,tr[i]==NDF),
      ForAll(j,
        Implies(j>=i,tr[j]==NDF)
      )
    )
  )
\end{Verbatim}
\end{mdframed}
\vspace{-10pt}
\caption{Encode constraints for {\tt tr:Trace} in Z3py}
\label{fig:tr_constraint_case1}
\end{center}
\end{figure}



\begin{itemize}
\item
The first constraint (line 4) says that the all indexes of {\tt tr} below 1 should be undefined because in VDM-SL, a sequence's index starts from 1.
\item
The second constraint (lines 6-13) says that if an index is defined, i.e., not undefined, all indexes lower than it should be all defined.
\item
The third constraint (line 15-22) says tat if an index is undefined, all indexes higher than it should be all undefined.
\end{itemize}

From the above three constraints, we limited an array ({\tt tr}) to a sequence that starts from index 1 to its last element. Now we have encoded both types {\tt nat1} and {\tt Trace} used in PO8 of the VDM-SL model. However, before proceed to encode the obligation, it is necessary to encode the function {\tt len} that computes the length of a sequence. As shown in Fig.~\ref{fig:tr_len_case1}, we defined an uninterpreted function {\tt len\_tr} for {\tt tr}, the instance of {\tt Trace}. Line 1 is the type declaration of {\tt len\_tr} which is an uninterpreted function accepting the input of a {\tt Trace} type and returning an integer type. Similar to defining constraints for {\tt tr:Trace}, we gave {\tt len\_tr} the constraint that consists of two cases: the length of a sequence can only be equal or greater than 0.

\begin{figure}[t]
\begin{center}
\begin{mdframed}[roundcorner=5pt,shadow=true]
\begin{Verbatim}[fontsize=\small,numbers=left]
len_tr=Function('len_tr',Trace,IntSort())

Or(
  And(
    len_tr(tr)==0,
    tr[1]==NDF
  ),
  And(
    len_tr(tr)>0,
    tr[len_tr(tr)]!=NDF,
    tr[len_tr(tr)+1]==NDF
  )
)
\end{Verbatim}
\end{mdframed}
\vspace{-10pt}
\caption{Encode {\tt len\_tr} for {\tt tr:Trace} in Z3py}
\label{fig:tr_len_case1}
\end{center}
\end{figure}

\begin{enumerate}
\item
If {\tt len\_tr} returns 0, then {\tt tr} is an sequence with no element. That is, {\tt tr[1]} has to be undefined so that all indexes of {\tt tr} are then undefined based on previous constraints of {\tt tr}. (lines 4-7)
\item
If {\tt len\_tr} returns an integer greater than 0, then {\tt tr} has defined elements till its length index {\tt len\_tr(tr)} such that the element at and after index {\tt len\_tr(tr)+1} should be undefined. (line 8-12)
\end{enumerate}

At this point, we have defined types and constraints needed for encoding PO8. According to the steps described in Section~\ref{section:encode-strategy}, we now proceed to the next step: obligation negation. The negated PO8 is shown in Fig.~\ref{fig:po8_negation_case1}. Compare to the original PO8 in Fig.~\ref{fig:po8_case1}, the negated PO8 also eliminated the forall quantifier so that we do not need to use forall in Z3py encoding. The PO can be recognized as the formula $\exists~[tr:Trace,aperi:nat1,vdel:nat1,i:nat] P \land Q \land \neg R$ where $P$, $Q$, and $\neg R$ can be encoded separately.

\begin{figure}[t]
\begin{center}
\begin{mdframed}[roundcorner=5pt]
\begin{Verbatim}[fontsize=\small]
Exists tr:Trace, aperi:nat1, vdel:nat1 &
  (Exists i in set (inds (tl tr)) &
     Not(
      ((i mod aperi) = (vdel + 1)) =>
       (i in set (inds tr))
     )
  )
\end{Verbatim}
\end{mdframed}
\vspace{-10pt}
\caption{Obligation negation of PO8}
\label{fig:po8_negation_case1}
\end{center}
\end{figure}


\begin{figure}[t]
\begin{center}
\begin{mdframed}[roundcorner=5pt,shadow=true]
\begin{Verbatim}[fontsize=\small,numbers=left]
  And(aperi>=1,vdel>=1)        #nat1
  And(i>=1, i<=len_tr(tr)-1)   #P
  i%aperi==vdel+1              #Q
  Not(And(i>=1,i<=len_tr(tr))) #not R
\end{Verbatim}
\end{mdframed}
\vspace{-10pt}
\caption{Encoded formulas of PO8 in Z3py}
\label{fig:po8_encoded_case1}
\end{center}
\end{figure}

Finally, we encoded the negated PO8 as the Z3py code shown in Fig.~\ref{fig:po8_encoded_case1}. The encoding has four constraints: two {\tt nat1} variables (line 1); encoding of $P$ ({\tt i in set (inds (tl tr)}) that is actually indexes from 1 to {\tt len tr -1} (line 2); encoding of $Q$ ({\tt (i mod aperi) = (vdel + 1)}) (line 3); encoding of $\neg R$ ({\tt not(i in set (inds tr))}) (line 4). Note that {\tt hd} and {\tt tl tr} are operators of the head and tail of a sequence, and {\tt inds} is the operator of gathering indexes of defined elements of a sequence and returns a set of natural numbers. Also note that in the encoding in Fig.~\ref{fig:po8_encoded_case1}, we did not explicitly encode {\tt hd}, {\tt tl}, and {\tt inds} and simplified the results of these operators as constraints related to ranges (upper and lower bound) of {\tt tr}. The encoded PO8 (negated) was checked by the solver object in Z3py for the satisfiability. The result was {\tt unsat}, which means that PO8 is proved.

% With the above two cases of {\tt len\_tr} that defines the length function of traces, together with constraints of {\tt tr}, and instances of types {\tt nat1}, we encoded PO8 as shown in  

% The data types used in the model are defined as follows:

% \begin{mdframed}[roundcorner=5pt]
% \begin{Verbatim}[fontsize=\small]
%   Trace = seq of [Event];
%   Event = <A> | <V>;

%   state Pacemaker of
%     aperiod : nat 
%     vdelay  : nat
%   init p == p = mk_Pacemaker(15,10)
%   end
% \end{Verbatim}
% \end{mdframed}

% {\tt Trace} is of sequence type of {\tt Event}, where {\tt Event} is of quote type with two values {\tt <A>} and {\tt <V>}. The state is defined as two natural numbers with initial value of {\tt (15,10)}. In Z3, natural numbers is encoded as positive integers (integers with constraint of greater and equal to zero); quote type can be encoded as enumeration type {\tt EnumSort}; sequence type can be encoded as {\tt ArraySort}. The definitions of {\tt Event} and {\tt Trace} are as follows:


% Furthermore, since {\tt Trace} is defined as an array in Z3, we added constraints that limit an array as a sequence. Note that in Z3, the constraints are applied on each instance of type {\tt Trace} ({\tt tr} in the following formulas), not on the type itself.


% $(len\_tr(tr) = 0 \land tr[1] = 0) \lor (len\_tr(tr) > 0 \land tr[len\_tr(tr)] \neq NDF \land tr[len\_tr(tr)+1] = NDF$


% {\tt len\_tr} is defined as an uninterpreted function that takes {\tt tr} as input and returns an integer indicates the length of {\tt tr}. Since the length of a sequence can only be equal or greater than 0, the constraint of {\tt len\_tr} is divided into two cases:


% We have defined the context information related to type {\tt Trace}. Now we can procceed to encode and prove the POs of the model. For this VDM-SL model, Overture generated nine POs. Most of the POs are similar so we only demonstrate PO8 and PO3.

% $\bullet$ PO8 is a legal sequence application obligation of operation {\tt Pace} which is explicitly specified. This PO requires that the formulas specified with trace {\tt tr} be legal, i.e., computable to get its result.



% In PO8 showed above, {\tt tl tr} is the tail of {\tt tr}, and {\tt inds} is the operator of gathering indexes of defined elements of {\tt tr} as the set of natural numbers. To prove PO8, firstly we took negation of PO8 which results a quantifier eliminated formula.


% Note that the {\tt oldstate} is not used in the operation so that we remove it in the above formula. 
% To encode the PO in Z3, the PO can be recognized as the formula $\exists~[tr:Trace,aperi:nat1,vdel:nat1,i:nat] P \land Q \land \neg R$ where $P$, $Q$, and $\neg R$ can be encoded separately.


% {\tt tl tr} is the tail of {\tt tr}, and {\tt inds} is the operator of gathering indexes of defined elements of {\tt tr} as the set of natural numbers.
% In the encoding of $P$, the tail related formula can be encoded as in the range of $[1,(len~tr)-1]$ so that it is not required to define {\tt tl} and {\tt inds} for sequences. Finally, we checked the satisfiability of the encoded Z3 code and got {\tt unsat}, which means that the PO is proved since we have negated it.


% {\tt Periodic}: legal sequence application obligation\\
% This PO is meant to represent the applicability of statements used in the function {\tt Periodic}, that is, the expressions in {\tt Periodic} are computable to get a result. The formula of the PO is in the form of 
% $\forall tr,e,p. ~\forall t \in inds ~tr. ~(~P \rightarrow ~(~Q \rightarrow ~(~R \rightarrow ~(~S \rightarrow ~(~\forall i. ~T \rightarrow U~)~)~)~)$. where 
% \begin{itemize}
% \item
% $tr: Trace$;~$e:Event$;~$p:nat1$;~$t:nat$
% \item
% $P = t \in set~(inds~tr)$
% \item
% $Q = (tr(t)=e)$
% \item
% $R = (t+p) ~<=~ (len~tr)$
% \item
% $S = (tr((t+p))=e) ~\land~ \forall i \in \{(t+1, \ldots, (t+p)-1\}. ~tr(i) \neq e )$
% \item
% $T = ((t+p) > (len~tr)~)$
% \item
% $U = (\forall i \in \{(t+1), \ldots, len~tr \}. ~i \in set (inds~tr)$
% \end{itemize}


% After taken negation of the formula, we got 
% $\exists tr,e,p,t,i. ~(P \land Q \land Q \land S \land T \land \neg U~)$. In the negated case, the expected result of the negated PO is unsatisfiable, i.e., unsat. The

\subsubsection{Proof Obligation 3}

PO3 of the abstracted pacemaker model is the postcondition satisfiable obligation of the implicitly specified operation {\tt FaultHeart} that generates a trace of heart pulse signals. PO3 says that the postcondition of {\tt FaultHeart} has to be satisfiable so that the specification of {\tt FaultHeart} is implementable. Fig.~\ref{fig:po3_case1} shows PO3 along with the specification of state and operation {\tt FaultyHeart}.

% \medskip
\begin{figure}[t]
\begin{center}
\begin{mdframed}[roundcorner=5pt]
\begin{Verbatim}[fontsize=\small,numbers=left]
exists tr:Trace &
  post_FaultyHeart(oldstate, tr, newstate)

state Pacemaker of
    aperiod : nat 
    vdelay  : nat
  init p == p = mk_Pacemaker(15,10)
  end

FaultyHeart() tr : Trace
post len tr = 100 and
  Periodic(tr,<A>,aperiod) and 
  not Periodic(tr,<V>,aperiod);
\end{Verbatim}
\end{mdframed}
\vspace{-10pt}
\caption{PO3, State {\tt Pacemaker}, and operation {\tt FaultyHeart} of Abstraced Pacemaker}
\label{fig:po3_case1}
\end{center}
\end{figure}
% \medskip

The state {\tt Pacemaker} is specified as a pair of natural numbers {\tt aperiod} and {\tt vdelay}. The postcondition of {\tt FaultyHeart} is not specified with {\tt oldstate} which has suffixed tilde, i.e, {\tt aperiod\textasciitilde} and {\tt vdelay\textasciitilde}. Therefore, PO3 is a simplified postcondition satisfiable obligation saying that there exists a {\tt Trace} {\tt tr} which satisfies the postcondition specified with only {\tt newstate}\footnote{How the {\tt newstate} is calculated from {\tt oldstate} is not specified in {\tt FaltyHeart}.}. The postcondition of {\tt FaultyHeart} calls the function {\tt Periodic} with three inputs. {\tt Periodic} shown below is an explicitly specified function and returns a boolean value. Therefore, state {\tt Pacemaker}, postcondition of {\tt FaultyHeart} ({\tt post\_FaultyHeart}, and function {\tt Periodic} are the context information of PO3. 

\begin{mdframed}[roundcorner=5pt]
\begin{Verbatim}[fontsize=\small]
Periodic: Trace * Event * nat1 -> bool
Periodic(tr,e,p) ==
  forall t in set inds tr &
   (tr(t) = e) =>
   (t + p <= len tr =>
   ((tr(t+p) = e and
     forall i in set {t+1, ..., t+p-1} &
       tr(i) <> e)) and
    (t + p > len tr =>
     forall i in set {t+1, ..., len tr} &
       tr(i) <> e));
\end{Verbatim}
\end{mdframed}

Though the specification of {\tt Periodic} is relatively long, the encoding of {\tt Periodic} is similar to the encoding of PO8. It is easy to observe similar computations on the trace {\tt tr} such as {\tt in set inds tr} and {\tt len tr}, and we can encode them in the same way as handling PO8. Note that the type {\tt Trace} is again needed in proving PO3 so that the encoding in PO8 above can be used as constraints here. In the case of PO3, we did not apply negation to the formula since existence quantifier in the prefix of a formula is preferred in Z3. As a result, we got a result of {\tt sat} along with a model that satisfies PO3. Therefore, PO3 was proved.

The results of all nine POs of the Abstracted Pacemaker VDM-SL model is showed in Table~\ref{tbl:result1} with information of whether negation is applied, and the time used.

\begin{table}[htb]
\begin{center}
\begin{tabular}{|c|c|r|r|}
\hline
PO\#	&	negated	&	result	&	time (sec.) \\ \hline
1		&	Y		&	sat		&	0.031 \\ \hline
2		&	Y		&	unsat	&	0.031 \\ \hline
3		&	N		&	sat		&	15.109 \\ \hline
4		&	Y		&	unsat	&	0.032 \\ \hline
5		&	Y		&	unsat	&	0.046 \\ \hline
6		&	Y		&	unsat	&	0.048 \\ \hline
7		&	Y		&	unsat	&	0.062 \\ \hline
8		&	Y		&	unsat	&	0.031 \\ \hline
9		&	Y		&	sat		&	0.047 \\ \hline
\end{tabular}
\end{center}
\caption{Absract Pacemaker Result}
\label{tbl:result1}
\end{table}


\subsection{Telephone Exchange}

This model specifies an abstracted telephone exchange system. In this model, the operations specify the events which can be initiated either by the system or by a subscriber (user) with implicit style. The system state monitors the calling status of users and the connecting status among users. Type in the model are based on quote types used to indicate the discrete states of users, then maps are specified to relate the calling and connecting status to users as the system state. An invariant is specified for the state {\tt Exchange} to declare the constraints among users and their status in the system.

\begin{mdframed}[roundcorner=5pt]
\begin{Verbatim}[fontsize=\small]
Subscriber = token;
Initiator =  <AI> | <WI> | <SI>;
Recipient = <WR> | <SR>;
Status = <fr> | <un> |
         Initiator | Recipient;
                                                                      
state Exchange of
  status: map Subscriber to Status
  calls:  inmap Subscriber to Subscriber
inv mk_Exchange(status, calls) == 
  forall i in set dom calls & 
    (status(i) = <WI> and
     status(calls(i)) = <WR>) or
    (status(i) = <SI> and
     status(calls(i)) = <SR>)
init s == s = mk_Exchange({|->},{|->})
end
\end{Verbatim}
\end{mdframed}

The strategy of encoding the above types is (1) applying user-defined types in Z3 to encode the quote types; (2) applying uninterpreted functions to encode map type. Here we only demonstrate a few selected types since the others are similar. 

\begin{mdframed}[roundcorner=5pt,shadow=true]
\begin{Verbatim}[fontsize=\small]
Recipient = Datatype('Recipient')
Recipient.declare('WR')
Recipient.declare('SR')
Recipient = Recipient.create()

Status = Datatype('Status')
Status.declare('fr')
Status.declare('un')
Status.declare('INITIATOR',
               ('get_initiator', Initiator))
Status.declare('RECIPIENT',
               ('get_recipient', Recipient))
Status = Status.create()

Status_lift = Datatype('Status_lift')
Status_lift.declare('STATUS',
                    ('get_status', Status))
Status_lift.declare('NDF')
Status_lift = Status_lift.create()

status = Function('status',
                  Subscriber,Status_lift)
calls = Function('calls',
                 Subscriber,Subscriber_lift)
\end{Verbatim}
\end{mdframed}

From above code, the type {\tt Recipient} is defined as an user-defined type with values {\tt WR} and {\tt SR}. Type {\tt Status} is further defined as a user-defined type including {\tt Recipient} and {\tt Initiator}. Note that since VDM uses partial functions, we also defined a lifted type of {\tt Status}. Though {\tt Subscriber} is of token type which does not have specific values when declared, we treated token types as quote types with predefined values such as {\tt S1}, {\tt S2}, and so on. Finally the two maps {\tt status} and {\tt calls} are defined as uninterpreted functions. Overture generated 27 POs for this model. Here we present the encoding and proving of PO1 and PO14.

$\bullet$ PO1 is a legal map application obligation that checks whether maps in the system state are applicable. 

\begin{mdframed}[roundcorner=5pt]
\begin{Verbatim}[fontsize=\small]
forall
  mk_Exchange(status, calls):EXCH`Exchange &
    (forall i in set (dom calls) &
      (i in set (dom status))
    )
)
\end{Verbatim}
\end{mdframed}

Since PO1 explicitly uses the system state, we need to consider the invariant of the system state as the context information of PO1. 
Instead of demonstrate the full encoding of the invariant, we only demonstrate the key encoding for {\tt i in set (dom calls)} and {\tt i in set (dom status)}. Though there is set inclusion used, instead of defining a set, we can encode the set inclusion as whether {\tt calls(i)} is defined or not.

\begin{mdframed}[roundcorner=5pt,shadow=true]
\begin{Verbatim}[fontsize=\small]
  calls(i) != Subscriber_lift.NDF
  status(i)!= Status_lift.NDF
\end{Verbatim}
\end{mdframed}

For PO1, we applied negation to the formula so that the {\tt forall} quantifier for maps {\tt calls} and {\tt status} is removed. The result was {\tt unsat} which means PO1 was proved.

\begin{mdframed}[roundcorner=5pt,shadow=true]
\begin{Verbatim}[fontsize=\small]
Not(
  ForAll(i,
    Implies(
      calls(i)!=Subscriber_lift.NDF,
      status(i)!=Status_lift.NDF
    )
  )
)
\end{Verbatim}
\end{mdframed}

$\bullet$ PO14 is an enumeration map injectivity obligatoin of the operation {\tt Answer} of the system. PO14 has free variables, {\tt r} and {\tt status}, in its formula. These free variables are treated as the quantifier {\tt forall} is prefixed. There are also intermediate variables in the formula such as {\tt m1}, {\tt m2}, {\tt d3}, and {\tt d4}. Note that the inverse function of a map, and the range restriction function {\tt :>} are {\tt applied} on calls and {\tt status}. Also note that {\tt calls} is a one-to-one map {\tt inmap}.

\begin{mdframed}[roundcorner=5pt]
\begin{Verbatim}[fontsize=\small]
(r in set (dom (status :> {<WR>}))) =>
  (forall m1, m2 in 
    set {{r |-> <SR>}, 
         {(inverse calls)(r) |-> <SI>}} &
    (forall d3 in set (dom m1),
            d4 in set (dom m2) &
      ((d3 = d4) => (m1(d3) = m2(d4)))
))
\end{Verbatim}
\end{mdframed}

We negated PO14 so that the {\tt forall} quantifiers are removed. To encode the negated formula of PO14, similar technique for  encoding {\tt set (dom status)} for PO1 can be applied. For the range restriction function, since the range is a singleton set with element value of {\tt <WR>}, we can skip encoding set notations while limiting the map value {\tt status(r)} to {\tt <WR>}. Thus, {\tt r in set (dom (status :> \{<WR>\}))} is encoded as follows.

\begin{mdframed}[roundcorner=5pt,shadow=true]
\begin{Verbatim}[fontsize=\small]
status(r)==
  Status_lift.STATUS(
    Status.RECIPIENT(Recipient.WR) )
\end{Verbatim}
\end{mdframed}

For the map inverse function, we chose not to define a new map as the inverse of {\tt calls}, but encoded {\tt m1 = \{(inverse calls)(r) |-> <SI>\}} as a case of {\tt m1 in set \{\{r |-> <SR>\}, \{(inverse calls)(r) |-> <SI>\}\}}. The case was encoded as the collection of several conditions. 

\begin{mdframed}[roundcorner=5pt,shadow=true]
\begin{Verbatim}[fontsize=\small]
# inverse map is not empty
calls(i)!=Subscriber_lift.NDF 
# inverse_calls(r) == i
Subscriber_lift.get_subscriber(calls(i))==r

m1(i)==Status_lift.STATUS(
         Status.INITIATOR(Initiator.SI)),

ForAll(k, 
  Implies(k!=i, m1(k)==Status_lift.NDF) )
\end{Verbatim}
\end{mdframed}

In above code, the first two conditions promise that (1) the part of map, where key {\tt i} is used and inverse is applied, has to be defined; (2) the domain of the inversed map, i.e., the range of the original map, is {\tt r}. Then {\tt m1(i)} where key {\tt i} is referred should be {\tt <SI>}. Finally the last condition guarantees that {\tt m1} should be undefined for all indexes except {\tt i}.

Recall that {\tt calls} is defined as {\tt inmap}, a one-to-one map. Thus, we had to add  the conditions about {\tt inmap} as context information: for any two different users, either both users calls no one (undefined) or they are calling different users.

\begin{mdframed}[roundcorner=5pt,shadow=true]
\begin{Verbatim}[fontsize=\small]
k  = Const('k',Subscriber)
l  = Const('l',Subscriber)
ForAll([k,l],
  If(
    k==l,
    calls(k)==calls(l),
    Or(
      And(
        calls(k)==Subscriber_lift.NDF,
        calls(l)==Subscriber_lift.NDF
      ),
      calls(k)!=calls(l)
    )
  )
)
\end{Verbatim}
\end{mdframed}
\medskip

The result of checking PO14 (negated) ware expected to be {\tt unsat}, but we got {\tt sat}. Z3 returned a model that satisfies the negated formula and constraints so that we could examine where is the problem. We found that {\tt calls} in the returned model is a map where users calls themselves. This does not violate the type of {\tt inmap}, but an user cannot call him/her-self in a telephone exchange system. Therefore, the checking of PO14 suggested that the VDM-SL model should add a constraint on {\tt calls}. As an example of correction, we may add the constraint below as the type invariant of {\tt calls} or the additional state invariant\footnote{We have added the constraint and checked again, and the PO was proved successfully.}.

\begin{mdframed}[roundcorner=5pt]
\begin{Verbatim}[fontsize=\small]
forall i in set dom calls & calls(i) != i
\end{Verbatim}
\end{mdframed}

In this case study, we did not check all the 27 POs since the encoding could not cover all POs. The main issue is that the maps are encoded as uninterpreted functions so that {\tt forall} quantifier cannot be applied. This results that we could not check the postcondition satisfiable obligations of this model\footnote{Besides postcondition satisfiable obligations, there are two POs with type check function {\tt is\_(name, type)}, for which Z3 encoding is not available and then skipped.}. Finally, we checked 16 out of all 27 POs.

\subsection{Discussion}

% Overture generated 27 POs for this model. These POs can be divided into several groups as showed below. In this case study we encoded and checked the first three groups using the above encoding of types. The reason of not handling the postcondition obligations is that the maps defined using uninterpreted functions cannot be used together with quantifiers. Also, map inverse obligations have {\tt is\_(name, type)}, the function of type check provided by VDM, in their formulas. There is no appropriate encoding for such type check functions in Z3. 

% \begin{enumerate}
% \item
% Legal map application obligations, invariant satisfiable obligations: 1 to 7.
% \item
% Enumeration map injectivity obligations: 9, 14, 18, 21, 25.
% \item
% Legal map application obligations with inverse map: 12, 17, 20, 23.
% \item
% Operation postcondition satisfiable obligations: 8, 10, 11, 15, 16, 19, 22, 26, 27
% \item
% Map inverse obligations: 13, 24.
% \end{enumerate}

Points of discussion:
(1) Automation: Is the encoding can be automated? partly or fully? 
(2) Patterns: What patterns are repeated used in encoding?

