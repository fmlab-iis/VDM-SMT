% General information about VDM
The Vienna Development Method (VDM)~\cite{Jones:1990:SSD:94062,Fitzgerald:2005:VDO:1044891,Fitz:PGLarson:2009book} is a formal method which supports modeling and analysis of software systems at various levels of abstraction. A VDM specification, i.e. software specifications described in a VDM model, uses a combination of implicit and/or explicit definitions of functionalities to describe software specifications acquired from software requirements. VDM has a strong record of applications for design and specification of software systems in industry~\cite{Larsen:2007:RIA:2227886.2227894,DBLP:conf/fm/KuritaCN08,DBLP:journals/ijsi/KuritaN09}. The well-known basic benefits of using VDM to describe software specifications are from the accuracy and unambiguity of VDM, which is common for other formal methods like B. 

% Validation methods of VDM
To validate a VDM model, existing tools such as Overture~\cite{Larsen:2010:OII:1668862.1668864} and VDMTools provide static consistency check regarding the syntax and type constraints of VDM specifications. The semantic validation of VDM specifications are rely on proof obligations (POs)~\cite{AL:97:POGV} generated by the above tools, and theorem provers are applied to discharge the generated POs. Testing (specification animation) by running a VDM model with an interpreter~\cite{Prehn:1991:LNCS551} is the alternative way of validation. This requires that the VDM model to be specified explicitly so that an interpreter can produce specific values.

% More about proof obligations and the motivation: solve as many as possible POs

In this paper, we propose an alternative approach to discharging the POs of VDM specifications using SMT solvers. Instead of translating the whole VDM model, we encode each POs whith its context information such type constraints involved, then release the encoded formulas to prove the PO. The advantages of proposed approach are (1) the encoding involves only a few segments of a VDM specification; (2) the proof process is automated with SMT solvers; (3) If a PO's proof fails, a counterexample model is returned for further examination. 

More specifically, in this approach we encode and prove POs of VDM-SL models as solving SMT formulas with Python API of Z3~\cite{MB:08:ZSS}. Z3 is one of the popular SMT solvers widely used in software verification and its Python API provides flexibility in building SMT formulas than SMT-LIB. We have conducted some case studies the results showed that our approach can discharge the majority of POs of a VDM-SL model with efficiency. 

The structure of this paper is as follows: blah blah blah.

Main story: To validate VDM models, discharging POs is the major way. However, the POs of a VDM model may rise to hundreds [give an example LOC of model vs. number of POs]. Because of undecidability of VDM POs, discharging all POs is very difficult. Our approach aims to release POs as many as possible by introducing SMT solvers. We chose Z3 because it is one of the most popular SMT solver. The Python API is handy for flexible encoding. The results are good.