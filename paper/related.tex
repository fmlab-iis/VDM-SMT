% Development history of VDM POG.
Proof is an essential to guarantee the consistency of VDM models~\cite{978-3-540-19813-0}, and the proof obligation generation of VDM is firstly proposed by B. K. Aichernig and P. G. Larsen~\cite{AL:97:POGV}. Later, A. Ribeiro and P. G. Larsen worked on proof obligation generation and discharging related to termination of recursive functions~\cite{Ribeiro2010}.

% Work related to discharging POs using theorem provers. 
There are several researches focusing on translation from VDM to theorem provers and discharging POs of VDM with theorem provers. S. Agerholm~\cite{Agerholm1996} proposed translation and from VDM-SL to PVS. This work proposed a translation from a subset of VDM-SL to PVS for type checking. S. Maharaj and J. Bicarregui~\cite{632849} used Agerholm's translation to assist the verification of VDM-SL models and refinements. Later, S. D. Vermolen et al.~\cite{Verm:2007:master,Vermolen:2010:PCV:1774088.1774608} utilized the parsing mechanism of Overture tools and developed automated translation from a subset of VDM-SL to HOL theorem prover. Vermolen's work also provided some tactics to support automated discharging of POs. L. D. Cuto et al.~\cite{CFP:14:TVCSAP} used the proof obligation generation mechanism for translating CML, a modeling language combing VDM-SL and CSP, to Isabella/HOL for automated proof. Recently, work on improvement of translating VDM-SL to HOL are also proposed~\cite{CT:15:EOCGTIS}.

% Comparison to our approach
The theorem prover based approaches has common issues: (1) the translation from VDM-SL to theorem provers may be erroneous since the translation applies on the whole VDM-SL model; (2) The proof process is usually complicated and requires an expert; (3) if theorem provers fail to prove a PO, there is little information to indicate that the problem is due to incorrect VDM-SL model, incorrect translation, or incorrect tactics. Compare to our approach that discharges VDM-SL proof obligations with SMT solvers, we adopt the efficiency of SMT solvers and our encoding does not require to be applied on the whole VDM-SL model. Furthermore, the functionality of counterexample generation is helpful to determine where the problem is. However, encoding in our approach is manual at this point and still need improvements.

% Work related to validation/verfication of VDM models with model checking.
On the other hand, verification and validation of VDM-SL models with model checking techniques are recently proposed. K. Lausdahl~\cite{kenneth:ifm2013} proposed a semantics-preserving translation that constructs an Alloy model from a subset of VDM-SL. This work aims to support the validation of implicitly specified VDM-SL models with Alloy. H-H. Lin et al.~\cite{DBLP:conf/ftscs/LinOKA15} proposed an approach that combines SPIN model checker and the VDMJ interpreter for verifying explicitly specified VDM-SL models. Lausdahl's work is similar to our approach since Alloy is based on SAT solvers, but SMT solvers in our approach can solve formulas more efficiently. Both the two works require types involved in a VDM-SL model being bounded, while bounds are not necessary for the encoding in our approach. Also, there are issues for Alloy and SPIN to represent the rich and complex VDM-SL types.

% For example, \cite{Verm:2007:master,Vermolen:2010:PCV:1774088.1774608} translate VDM models to HOL and then prove the POs within the environment of HOL. However, this approach has several issues: (1) the translation is applied to the whole VDM model so that it may be tedious and error-prone; (2) The proof process is usually complicated and requires an expert; (3) It is not clear where the problem is when the proof fails.

% \cite{CFP:14:TVCSAP,CT:15:EOCGTIS}