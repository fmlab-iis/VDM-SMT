
%% bare_conf.tex
%% V1.4b
%% 2015/08/26
%% by Michael Shell
%% See:
%% http://www.michaelshell.org/
%% for current contact information.
%%
%% This is a skeleton file demonstrating the use of IEEEtran.cls
%% (requires IEEEtran.cls version 1.8b or later) with an IEEE
%% conference paper.
%%
%% Support sites:
%% http://www.michaelshell.org/tex/ieeetran/
%% http://www.ctan.org/pkg/ieeetran
%% and
%% http://www.ieee.org/

%%*************************************************************************
%% Legal Notice:
%% This code is offered as-is without any warranty either expressed or
%% implied; without even the implied warranty of MERCHANTABILITY or
%% FITNESS FOR A PARTICULAR PURPOSE! 
%% User assumes all risk.
%% In no event shall the IEEE or any contributor to this code be liable for
%% any damages or losses, including, but not limited to, incidental,
%% consequential, or any other damages, resulting from the use or misuse
%% of any information contained here.
%%
%% All comments are the opinions of their respective authors and are not
%% necessarily endorsed by the IEEE.
%%
%% This work is distributed under the LaTeX Project Public License (LPPL)
%% ( http://www.latex-project.org/ ) version 1.3, and may be freely used,
%% distributed and modified. A copy of the LPPL, version 1.3, is included
%% in the base LaTeX documentation of all distributions of LaTeX released
%% 2003/12/01 or later.
%% Retain all contribution notices and credits.
%% ** Modified files should be clearly indicated as such, including  **
%% ** renaming them and changing author support contact information. **
%%*************************************************************************


% *** Authors should verify (and, if needed, correct) their LaTeX system  ***
% *** with the testflow diagnostic prior to trusting their LaTeX platform ***
% *** with production work. The IEEE's font choices and paper sizes can   ***
% *** trigger bugs that do not appear when using other class files.       ***                          ***
% The testflow support page is at:
% http://www.michaelshell.org/tex/testflow/



\documentclass[conference]{IEEEtran}
% Some Computer Society conferences also require the compsoc mode option,
% but others use the standard conference format.
%
% If IEEEtran.cls has not been installed into the LaTeX system files,
% manually specify the path to it like:
% \documentclass[conference]{../sty/IEEEtran}





% Some very useful LaTeX packages include:
% (uncomment the ones you want to load)


% *** MISC UTILITY PACKAGES ***
%
%\usepackage{ifpdf}
% Heiko Oberdiek's ifpdf.sty is very useful if you need conditional
% compilation based on whether the output is pdf or dvi.
% usage:
% \ifpdf
%   % pdf code
% \else
%   % dvi code
% \fi
% The latest version of ifpdf.sty can be obtained from:
% http://www.ctan.org/pkg/ifpdf
% Also, note that IEEEtran.cls V1.7 and later provides a builtin
% \ifCLASSINFOpdf conditional that works the same way.
% When switching from latex to pdflatex and vice-versa, the compiler may
% have to be run twice to clear warning/error messages.






% *** CITATION PACKAGES ***
%
%\usepackage{cite}
% cite.sty was written by Donald Arseneau
% V1.6 and later of IEEEtran pre-defines the format of the cite.sty package
% \cite{} output to follow that of the IEEE. Loading the cite package will
% result in citation numbers being automatically sorted and properly
% "compressed/ranged". e.g., [1], [9], [2], [7], [5], [6] without using
% cite.sty will become [1], [2], [5]--[7], [9] using cite.sty. cite.sty's
% \cite will automatically add leading space, if needed. Use cite.sty's
% noadjust option (cite.sty V3.8 and later) if you want to turn this off
% such as if a citation ever needs to be enclosed in parenthesis.
% cite.sty is already installed on most LaTeX systems. Be sure and use
% version 5.0 (2009-03-20) and later if using hyperref.sty.
% The latest version can be obtained at:
% http://www.ctan.org/pkg/cite
% The documentation is contained in the cite.sty file itself.








% *** MATH PACKAGES ***
%
\usepackage{amsmath}
% A popular package from the American Mathematical Society that provides
% many useful and powerful commands for dealing with mathematics.
%
% Note that the amsmath package sets \interdisplaylinepenalty to 10000
% thus preventing page breaks from occurring within multiline equations. Use:
%\interdisplaylinepenalty=2500
% after loading amsmath to restore such page breaks as IEEEtran.cls normally
% does. amsmath.sty is already installed on most LaTeX systems. The latest
% version and documentation can be obtained at:
% http://www.ctan.org/pkg/amsmath





% *** SPECIALIZED LIST PACKAGES ***
%
\usepackage{algorithmic}
% algorithmic.sty was written by Peter Williams and Rogerio Brito.
% This package provides an algorithmic environment fo describing algorithms.
% You can use the algorithmic environment in-text or within a figure
% environment to provide for a floating algorithm. Do NOT use the algorithm
% floating environment provided by algorithm.sty (by the same authors) or
% algorithm2e.sty (by Christophe Fiorio) as the IEEE does not use dedicated
% algorithm float types and packages that provide these will not provide
% correct IEEE style captions. The latest version and documentation of
% algorithmic.sty can be obtained at:
% http://www.ctan.org/pkg/algorithms
% Also of interest may be the (relatively newer and more customizable)
% algorithmicx.sty package by Szasz Janos:
% http://www.ctan.org/pkg/algorithmicx




% *** ALIGNMENT PACKAGES ***
%
%\usepackage{array}
% Frank Mittelbach's and David Carlisle's array.sty patches and improves
% the standard LaTeX2e array and tabular environments to provide better
% appearance and additional user controls. As the default LaTeX2e table
% generation code is lacking to the point of almost being broken with
% respect to the quality of the end results, all users are strongly
% advised to use an enhanced (at the very least that provided by array.sty)
% set of table tools. array.sty is already installed on most systems. The
% latest version and documentation can be obtained at:
% http://www.ctan.org/pkg/array


% IEEEtran contains the IEEEeqnarray family of commands that can be used to
% generate multiline equations as well as matrices, tables, etc., of high
% quality.




% *** SUBFIGURE PACKAGES ***
%\ifCLASSOPTIONcompsoc
%  \usepackage[caption=false,font=normalsize,labelfont=sf,textfont=sf]{subfig}
%\else
%  \usepackage[caption=false,font=footnotesize]{subfig}
%\fi
% subfig.sty, written by Steven Douglas Cochran, is the modern replacement
% for subfigure.sty, the latter of which is no longer maintained and is
% incompatible with some LaTeX packages including fixltx2e. However,
% subfig.sty requires and automatically loads Axel Sommerfeldt's caption.sty
% which will override IEEEtran.cls' handling of captions and this will result
% in non-IEEE style figure/table captions. To prevent this problem, be sure
% and invoke subfig.sty's "caption=false" package option (available since
% subfig.sty version 1.3, 2005/06/28) as this is will preserve IEEEtran.cls
% handling of captions.
% Note that the Computer Society format requires a larger sans serif font
% than the serif footnote size font used in traditional IEEE formatting
% and thus the need to invoke different subfig.sty package options depending
% on whether compsoc mode has been enabled.
%
% The latest version and documentation of subfig.sty can be obtained at:
% http://www.ctan.org/pkg/subfig




% *** FLOAT PACKAGES ***
%
%\usepackage{fixltx2e}
% fixltx2e, the successor to the earlier fix2col.sty, was written by
% Frank Mittelbach and David Carlisle. This package corrects a few problems
% in the LaTeX2e kernel, the most notable of which is that in current
% LaTeX2e releases, the ordering of single and double column floats is not
% guaranteed to be preserved. Thus, an unpatched LaTeX2e can allow a
% single column figure to be placed prior to an earlier double column
% figure.
% Be aware that LaTeX2e kernels dated 2015 and later have fixltx2e.sty's
% corrections already built into the system in which case a warning will
% be issued if an attempt is made to load fixltx2e.sty as it is no longer
% needed.
% The latest version and documentation can be found at:
% http://www.ctan.org/pkg/fixltx2e


%\usepackage{stfloats}
% stfloats.sty was written by Sigitas Tolusis. This package gives LaTeX2e
% the ability to do double column floats at the bottom of the page as well
% as the top. (e.g., "\begin{figure*}[!b]" is not normally possible in
% LaTeX2e). It also provides a command:
%\fnbelowfloat
% to enable the placement of footnotes below bottom floats (the standard
% LaTeX2e kernel puts them above bottom floats). This is an invasive package
% which rewrites many portions of the LaTeX2e float routines. It may not work
% with other packages that modify the LaTeX2e float routines. The latest
% version and documentation can be obtained at:
% http://www.ctan.org/pkg/stfloats
% Do not use the stfloats baselinefloat ability as the IEEE does not allow
% \baselineskip to stretch. Authors submitting work to the IEEE should note
% that the IEEE rarely uses double column equations and that authors should try
% to avoid such use. Do not be tempted to use the cuted.sty or midfloat.sty
% packages (also by Sigitas Tolusis) as the IEEE does not format its papers in
% such ways.
% Do not attempt to use stfloats with fixltx2e as they are incompatible.
% Instead, use Morten Hogholm'a dblfloatfix which combines the features
% of both fixltx2e and stfloats:
%
% \usepackage{dblfloatfix}
% The latest version can be found at:
% http://www.ctan.org/pkg/dblfloatfix




% *** PDF, URL AND HYPERLINK PACKAGES ***
%
\usepackage{url}
% url.sty was written by Donald Arseneau. It provides better support for
% handling and breaking URLs. url.sty is already installed on most LaTeX
% systems. The latest version and documentation can be obtained at:
% http://www.ctan.org/pkg/url
% Basically, \url{my_url_here}.




% *** Do not adjust lengths that control margins, column widths, etc. ***
% *** Do not use packages that alter fonts (such as pslatex).         ***
% There should be no need to do such things with IEEEtran.cls V1.6 and later.
% (Unless specifically asked to do so by the journal or conference you plan
% to submit to, of course. )

% correct bad hyphenation here
\hyphenation{op-tical net-works semi-conduc-tor}

\usepackage{fancyvrb}
\usepackage[framemethod=tikZ]{mdframed}
\usetikzlibrary{shadows}
% \usepackage{tcolorbox}

\begin{document}
%
% paper title
% Titles are generally capitalized except for words such as a, an, and, as,
% at, but, by, for, in, nor, of, on, or, the, to and up, which are usually
% not capitalized unless they are the first or last word of the title.
% Linebreaks \\ can be used within to get better formatting as desired.
% Do not put math or special symbols in the title.
\title{Releasing VDM Proof Obligations with SMT Solvers}


% author names and affiliations
% use a multiple column layout for up to three different
% affiliations
\author{\IEEEauthorblockN{Hsin-Hung Lin}
\IEEEauthorblockA{Institute of Information Science\\
Academia Sinica\\
Email: hlin@iis.sinica.edu.tw}
\and
\IEEEauthorblockN{Bow-Yaw Wang}
\IEEEauthorblockA{Institute of Information Science\\
Academia Sinica\\
Email: bywang@iis.sinica.edu.tw}
}

% conference papers do not typically use \thanks and this command
% is locked out in conference mode. If really needed, such as for
% the acknowledgment of grants, issue a \IEEEoverridecommandlockouts
% after \documentclass

% for over three affiliations, or if they all won't fit within the width
% of the page, use this alternative format:
% 
%\author{\IEEEauthorblockN{Michael Shell\IEEEauthorrefmark{1},
%Homer Simpson\IEEEauthorrefmark{2},
%James Kirk\IEEEauthorrefmark{3}, 
%Montgomery Scott\IEEEauthorrefmark{3} and
%Eldon Tyrell\IEEEauthorrefmark{4}}
%\IEEEauthorblockA{\IEEEauthorrefmark{1}School of Electrical and Computer Engineering\\
%Georgia Institute of Technology,
%Atlanta, Georgia 30332--0250\\ Email: see http://www.michaelshell.org/contact.html}
%\IEEEauthorblockA{\IEEEauthorrefmark{2}Twentieth Century Fox, Springfield, USA\\
%Email: homer@thesimpsons.com}
%\IEEEauthorblockA{\IEEEauthorrefmark{3}Starfleet Academy, San Francisco, California 96678-2391\\
%Telephone: (800) 555--1212, Fax: (888) 555--1212}
%\IEEEauthorblockA{\IEEEauthorrefmark{4}Tyrell Inc., 123 Replicant Street, Los Angeles, California 90210--4321}}




% use for special paper notices
%\IEEEspecialpapernotice{(Invited Paper)}




% make the title area
\maketitle

% As a general rule, do not put math, special symbols or citations
% in the abstract
\begin{abstract}
The Vienna Development Method (VDM) is a formal method that supports modeling and analysis of software systems at various levels of abstractions. For a model specified by the VDM specification language (VDM-SL), the correctness of the model relies on discharging the proof obligations (POs), especially in the case of implicit specification. In this paper, we propose an approach that encodes and discharge POs of VDM-SL models using SMT solvers. More specifically, POs generated by the Overture tool are encoded and discharged in Z3 SMT solver using the Python API. Our case studies have shown that the approach can support discharging significant part of proof obligations of a VDM-SL model with efficiency.
\end{abstract}

% no keywords

% For peer review papers, you can put extra information on the cover
% page as needed:
% \ifCLASSOPTIONpeerreview
% \begin{center} \bfseries EDICS Category: 3-BBND \end{center}
% \fi
%
% For peerreview papers, this IEEEtran command inserts a page break and
% creates the second title. It will be ignored for other modes.
\IEEEpeerreviewmaketitle



\section{Introduction}
\label{section:introduction}

% General information about VDM
The Vienna Development Method (VDM)~\cite{Jones:1990:SSD:94062,Fitzgerald:2005:VDO:1044891,Fitz:PGLarson:2009book} is a formal method which supports modeling and analysis of software systems at various levels of abstraction. A VDM specification, i.e. software specifications described in a VDM model, uses a combination of implicit and/or explicit definitions of functionalities to describe software specifications acquired from software requirements. VDM has a strong record of applications for design and specification of software systems in industry~\cite{Larsen:2007:RIA:2227886.2227894,DBLP:conf/fm/KuritaCN08,DBLP:journals/ijsi/KuritaN09}. The well-known basic benefits of using VDM to describe software specifications are from the accuracy and unambiguity of VDM, which is common for other formal methods like B. 

% Validation methods of VDM
To validate a VDM model, existing tools such as Overture~\cite{Larsen:2010:OII:1668862.1668864} and VDMTools provide static consistency check regarding the syntax and type constraints of VDM specifications. The semantic validation of VDM specifications are rely on proof obligations (POs)~\cite{AL:97:POGV} generated by the above tools, and theorem provers are applied to discharge the generated POs. Testing (specification animation) by running a VDM model with an interpreter~\cite{Prehn:1991:LNCS551} is the alternative way of validation. This requires that the VDM model to be specified explicitly so that an interpreter can produce specific values for functions and operations.

% More about proof obligations and the motivation: solve as many as possible POs
Discharging POs of VDM models is essential to guarantee the consistency of the formal specifications. For a VDM model, the number of POs is usually large compare to the size of the model. For example, the sample VDM-SL model shown in Fig.~\ref{fig:module_sample} has less than thirty lines, and Overture generates nine POs for the model. In this paper, we propose an approach to discharging the POs of VDM specifications using SMT solvers. Here we focus on VDM-SL models and the objective is to discharge POs of an VDM-SL model as many as possible through SMT solvers. In this approach, we encode each POs whith its context information such type and state constraints, then release the encoded formulas to prove the PO. The advantages of our approach are (1) the encoding involves only a few segments of a VDM specification; (2) the proof process is automated with SMT solvers; (3) If a PO's proof fails, a counterexample model is returned for further examination. 

More specifically, in this approach we encode and prove POs of VDM-SL models as solving SMT formulas with Python API of Z3~\cite{MB:08:ZSS}. Z3 is one of the popular SMT solvers widely used in software verification and its Python API provides flexibility in building SMT formulas than using pure SMT-LIB~\cite{BarFT-SMTLIB}. We have conducted some case studies and the results showed that our approach can efficiently discharge the significant part of POs of a VDM-SL model. We chose the Overture tool as the PO generator in this paper\footnote{POs generated by Overture and VDMTools are slightly different in the numbers and styles of POs}.

The structure of this paper is as follows: Section~\ref{section:vdm} briefly introduce VDM with a sample VDM-SL example; Section~\ref{section:proof-obligations} describes the POs of VDM and explains the POs generated by the Overture tool; section~\ref{section:encode-strategy} describes our strategy of encoding VDM expressions with Python API of Z3; section~\ref{section:case-studies} demonstrates case studies with discussions; section~\ref{section:related-work} compares our approach to related work; section~\ref{section:conclusions} concludes this paper and points out the future work.

% Main story: To validate VDM models, discharging POs is the major way. However, the POs of a VDM model may rise to hundreds [give an example LOC of model vs. number of POs]. Because of undecidability of VDM POs, discharging all POs is very difficult. Our approach aims to release POs as many as possible by introducing SMT solvers. We chose Z3 because it is one of the most popular SMT solver. The Python API is handy for flexible encoding. The results are good.

% \section{Preliminaries}
% \label{section:preliminaries}

% 
\cite{MB:08:ZSS}

\section{VDM}
\label{section:vdm}

% Give a simple example with the syntax used in case studies and explain implicit and explicit specification in VDM-SL.

% General information of VDM
The Vienna Development Method (VDM) was originally developed in the 1970's at the IBM laboratories in Vienna~\cite{DBLP:conf/fm/1978}. The VDM Specification Language (VDM-SL) is a higher-order language with formally defined syntax and semantics~\cite{Prehn:1991:LNCS551,Larsen1995585}. VDM provides various abstract data types: basic types such as boolean, natural number, and token; advanced types such as record, product, set, and map. A state consists of typed variables. Typed variables in a state may be restricted by invariants and operations/functions may be specified with preconditions and postconditions.

% Explaining the sample VDM-SL model: part1
As a brief explanation of VDM-SL, Fig.~\ref{fig:module_sample} shows a sample VDM-SL model {\tt CMDS}. The specification is divided into several blocks: types (lines 1-4), state (lines 6-12), operations (lines 14-18), and functions (lines 20-27). {\tt CMD} is a union type of quotes which is usually used for enumeration. {\tt CMD} represents commands {\tt <R>} and {\tt <L>} denoting right and left. {\tt CMD\_series} is defined as a sequence type which is a list of commands. Note that the type of elements {\tt [CMD]} with square brackets means the union type {\tt CMD|nil}. We also defined a map type for counting the occurrences of commands. The state of the model is defined as a sequence of commands {\tt CMD\_series}. The state invariant specifies that adjacent elements of {\tt commands} have to be different. It is specified with {\tt forall} quantifier and {\tt len}, a function that returns the length of a sequence. Finally, the initial value of \texttt{commands} is defined to be an empty sequence.


% An example of VDM-SL model (or some segments of VDM-SL code) should have
% types: quote and union, sequence, token, map
% operations of sequence: len, hd, tl
% operations of map: range restriction
% operation of set: in set
% others: state def., forall statement
\begin{figure}[t]
\begin{center}
\begin{mdframed}[roundcorner=5pt]
\begin{Verbatim}[fontsize=\small,numbers=left]
types
CMD = <R> | <L>;
CMD_series = seq of [CMD];
CMD_times = map CMD to nat;

state S of
  commands : CMD_series
  inv s == forall k in set 
    {1,...,len s.commands - 1} &
    s.commands(k) <> s.commands(k+1)
  init p == p = mk_S([])
end

operations
  push_cmd(a:[CMD])
  pre commands = [] or hd commands <> a
  post hd commands = a and
       tl commands = commands~;

functions
  times_count : CMD_series -> CMD_times
  times_count(a) == {
    <R> |->
    len [ i | i in set inds a & a(i)=<R> ],
    <L> |->
    len [ i | i in set inds a & a(i)=<L> ]
  };
\end{Verbatim}
\end{mdframed}
\vspace{-10pt}
\caption{A sample VDM-SL specification: CMDS}
\label{fig:module_sample}
\vspace{-20pt}
\end{center}
\end{figure}

% Explaining the sample VDM-SL model: part2
The operation {\tt push\_cmd} is specified to accept a command {\tt CMD} or {\tt nil} and pushes the command as the head of {\tt commands}. The precondition demands that \texttt{commands} is an empty sequence or its head is different from the argument. The postcondition is specified using the old state, the state before the execution of {\tt push\_cmd}, that denoted as {\tt commands\textasciitilde} with postfixed tilde. The postcondition says that the old state is the tail of the new state {\tt commands}. Observe that {\tt push\_cmd} is an implicit operation that only specifies the signature with pre- and post-conditions. The function \texttt{times\_count} that counts the occurrences of commands is specified explicitly. Instead of using loop, the function uses sequence comprehension to construct the map of type {\tt CMD\_times}. Specifically, the resultant map has the domain $\{ \texttt{<R>}, \texttt{<L>} \}$. It maps $\texttt{<R>}$ to the length of the sequence which contains the indices of $\texttt{a}$ having the value $\texttt{<R>}$. That is, the resultant map gives the number of $\texttt{<R>}$ in the sequence $\texttt{a}$ on $\texttt{<R>}$. Simiarly, it gives the number of $\texttt{<L>}$ in $\texttt{a}$ on $\texttt{<L>}$. This also shows the expression power of VDM in specification. 

% tools and dialects of VDM (optional)
Existing tools such as Overture Tool~\cite{Larsen:2010:OII:1668862.1668864} and VDMTools~\cite{2008:VAS:1361213.1361214} provide graphical user interfaces for easy editing/building of VDM models. These tools also provide functionalities such as type/syntax checking and testing/animation (execution by the interpreter~\cite{Prehn:1991:LNCS551}) for validation and verification of VDM models. There are other dialects of VDM: VDM++ and VDM Real-Time (VDM-RT)~\cite{10.1007/11813040_11}. VDM++ is the extension of VDM-SL with object-orient concepts; VDM-RT further extends VDM++ with scheduling controls of threads or processes.


\section{Proof Obligations}
\label{section:proof-obligations}

% Objective: explain the semantics of Overture generated POs.
% Explain what is proof obligation in VDM. There are kinds of POs: domain, subtype, satisfiability, and termination. Any example?
% Explain POs generated by Overture tool. Any example?

% Generation information of POs
For a VDM model, a proof obligation (PO) is a statement that must be proved to ensure the consistency of the model. A PO contains a predicate with its context information as shown below. The context information is from code segments of the VDM model that relate to the predicate to be proved.

\begin{mdframed}[roundcorner=5pt]
\begin{Verbatim}[fontsize=\small]
PO: context information ==> predicate
\end{Verbatim}
\end{mdframed}

% Classifications of POs and sample POs from sample VDM-SL model
POs of VDM-SL can be classified type compatibility (subtype checking), domain checking, and satisfiability~\cite{AL:97:POGV,Vermolen:2010:PCV:1774088.1774608}. Type compatibility relates to types with invariant or subtypes; domain checking relates to the use of partial functions and partial operators; satisfiability relates postconditions of functions and operations. Both Overture and VDMTools generate POs with the syntax of VDM.

For the sample VDM-SL model shown in Fig.~\ref{fig:module_sample}, Overture generated ten POs. Here selected POs are demonstrated and explained. PO1 is called ``legal sequence application obligation'' that relates to the sequence involved expression (line 9-10) of the VDM-SL model {\tt CMDS}). PO1 says that for all k in the set from 1 to {\tt len s.commands -1}, k should always be within the bound of defined indices of the sequence {\tt s.commands}. Also, it can be recognized that if we use {\tt len s.commands} instead of {\tt len s.commands -1} the proof will fail since a sequence in VDM is indexed from 1 to its length. Also note that in PO1, the state {\tt CMDS'S} is mentioned, and we have to include the state invariant as well though it is not explicitly described in PO1.

\begin{mdframed}[roundcorner=5pt]
\begin{Verbatim}[fontsize=\small]
(forall s:CMDS`S & 
  (forall k in set 
    {1, ... ,((len (s.commands)) - 1)} & 
    (k in set (inds (s.commands)))))
\end{Verbatim}
\end{mdframed}

PO7 is an operation postcondition satisfiable obligation for {\tt push\_cmd} (line 15-18). PO7 says that for every input {\tt a:CMD} and {\tt oldstate}, if the precondition which takes inputs {\tt a} and {\tt oldstate} is satisfied, the postcondition which takes inputs {\tt a}, {\tt oldstate}, and {\tt newstate} should be satisfied. Note that instead of the specification of precondition, postcondition, and state in {\tt CMDS} module, special names {\tt oldstate}, {\tt newstate}, and {\tt post\_push\_cmd} are used. In VDM-SL, the {\tt pre\_} and {\tt post\_} prefix represent the precondition and postcondition of a function/operator separately. They are evaluable functions if given appropriate arguments, and return a boolean value. In a satisfiable obligation of an operation in which the state change is usually expected, the name of state (in this case {\tt commands}) represents {\tt newstate} while the same name postfixed with ``~\textasciitilde~'' (in this case {\tt commands\textasciitilde}) represents {\tt oldstate}.

\begin{mdframed}[roundcorner=5pt]
\begin{Verbatim}[fontsize=\small]
(forall a:[CMD], oldstate:CMDS`S & 
  (pre_push_cmd(a, oldstate) => 
    (exists newstate:CMDS`S & 
      post_push_cmd(a, oldstate, newstate))
))
\end{Verbatim}
\end{mdframed}


\section{Encode Strategy}
\label{section:encode-strategy}

% Explain the encode strategy: (1) explain Z3 and python API; (2) strategy on types; (3) take negation or not

% Z3 introduction, API introduction
As mentioned in Section~\ref{section:introduction}, our approach encodes and proves VDM POs with the Z3 SMT solver. Z3 is one of the popular SMT solver wildly used in verification of programs and software systems. Z3 has APIs for major programming languages such as C, Java, and Python. The APIs provide a better way of constructing SMT formulas than SMT-LIB~\cite{BarFT-SMTLIB} because one can adopt the flow control characteristics like loop and if-then-else of programming languages to build and solve formulas in a flexible and smart way. In our approach, we chose Z3's Python API (Z3py) as the encoding environment. 

% steps of encoding
% 1. check context information: types, functions, and operations involved in the obligation
% 2. encode context information: types and type invariants, functions, and operations (pre- and post-conditions if needed)
% 3. negation and forall quantifier elimination
Currently, the encoding is manually performed based on the obligation to be discharged and below are the steps of encoding and solving an obligation with SMT solvers.

\begin{enumerate}
\item
Determine the context information of the PO: The context information is the definitions specified in the VDM model that are involved in the obligation. The context information may include types, functions, and operations. Note that if a type is involved, its type invariant should be included, too.
\item
Encode context information: this step encodes the context information determined by the first step, and can be divided into several minor steps.
\begin{enumerate}
\item
For a VDM type in the context information, both the type itself and its invariant have to be encoded.
\item
For pre- and post-conditions of a function/operation specified in the VDM model, the encoding is needed if {\tt pre\_<name>} or {\tt post\_<name>} are used in the PO. For example, PO7 of the {\tt CMDS} module described in Section~\ref{section:proof-obligations} mentioned both precondition and postcondition, so both conditions shall be encoded.
\item
If a PO only considers a part of the specification of a type, function, or an operations, it is only needed to encode the expressions that are directly used in the PO. In other words, if a type, a function, or an operation is not directly mentioned, for example, a variable mentioned with {\tt name:type} style, or a function mentioned with {\tt fun\_name(arguments)} style, there is no need to encode the full specification of the type, the function, or the operation.
\end{enumerate}
\item
Obligation negation and quantifier elimination
\begin{enumerate}
\item
Obligation negation: In model checking with SMT solvers, the assertion to be checked is usually negated, and the result of unsatmeans the proof of the assertion succeeds. For the result of sat which means the assertion can be violated, SMT solvers return a model as the counterexample that shows an example of how the assertion is violated. In this work, we also take negation on the PO to be discharged. However, the elimination of forall quantifier should be considered together when applying the negation to the PO.
\item
Elimination of forall quantifiers: POs of VDM usually have forall quantifiers. To solve the formulas with forall quantifiers in Z3, though Z3 accepts formulas with forall quantifiers, there are limitations applying forall quantifiers in Z3 so that by default we have to eliminate forall quantifiers before encoding. In this work, forall quantifier usually causes difficulties in encoding and solving when complex types are involved. Obligation negation described above may achieve the elimination of troublesome foall quantifiers. On the other hand, obligation negation should be applied carefully not to introduce troublesome forall quantifiers from exists quantifiers.
\end{enumerate}
\item
Encode the predicate of obligation and check for satisfiability.
\end{enumerate}

% More details of encoding types, operators, and expressions
VDM has rich types from natural number and quote to sequence and maps. There are no direct matching for all VDM types to the types in Z3. For example, to encode a natural number, we need to declare an integer and then add a constraint of greater and equal to zero because there is no natural number type in Z3. The enumeration type in Z3 can be used to encode simple quote types. For complex types like sequences, maps, or records, we may use uninterpreted functions or arrays as the base type and capture the characteristic and operators of the VDM types by adding corresponding constraints. At this point, there is not yet a complete and systematic way of encoding VDM types in Z3. What we can do is to choose the encoding carefully based on the types and expressions involved in a PO. Based on our case studies, there are patterns of encoding that are frequently used for corresponding expression styles in VDM-SL models.

% Demonstrate Z3Py encoding
For more details, the Z3py encoding firstly needs to include the Z3 module and declare a solver object. We can add constraints of a PO to the solver object and solve the formulas for proof.

\begin{mdframed}[roundcorner=5pt,shadow=true]
\begin{Verbatim}[fontsize=\small]
  from z3 import *
  s = Solver()
  i = Int('i')
  s.add(i>=0)
\end{Verbatim}
\end{mdframed}

The above Python code imports the z3 module and create a solver object, then declare an integer {\tt i} with a constraint that {\tt i} is greater and equal to zero. In this case, {\tt i} is the encoding an instance of type {\tt nat} in VDM. Note that we cannot add constraints on types but on instance of types in Z3. This means that if there are several variables of the same VDM type that Z3 does not have, instances of the same number of variables have to be declared in Z3 with constraints for every instance added to the solver object. The steps and strategies of encoding discussed above will be demonstrated with details in Section~\ref{section:case-studies} through the report of case studies.


% Z3py encoding: taking negation and forall quantifier elimination
% Besides the encoding of VDM types, 
%  For example, if a VDM type is encoded as an uninterpreted function, the type cannot be used within forall quantifier. A simple way of quantifier elimination is negation the formula so the forall quantifier turns into exists quantifier which can be omitted in SMT formulas usually. 

\section{Case Studies}
\label{section:case-studies}

This section gives studies that apply our encoding to selected VDM-SL models and discusses the results. All the selected VDM-SL models are from the Overture repository\footnote{a url link here}.

\subsection{Abstract Pacemaker}

This model specifies an abstract pacemaker with some core functionalities of pulse trace selection\footnote{This model was developed by Sten Agerholm et al. in 1999 in connection with FM'99}. The data types used in the model are defined as follows:
% The model is described in VDM-SL as a short, flat specification. This enables abstraction from design considerations and ensures maximum focus on high-level, precise and systematic analysis

\begin{mdframed}[roundcorner=5pt]
\begin{Verbatim}[fontsize=\small]
  Trace = seq of [Event];
  Event = <A> | <V>;

  state Pacemaker of
    aperiod : nat 
    vdelay  : nat
  init p == p = mk_Pacemaker(15,10)
  end
\end{Verbatim}
\end{mdframed}

{\tt Trace} is of sequence type of {\tt Event}, where {\tt Event} is of quote type with two values {\tt <A>} and {\tt <V>}. The state is defined as two natural numbers with initial value of {\tt (15,10)}. In Z3, natural numbers is encoded as positive integers (integers with constraint of greater and equal to zero); quote type can be encoded as enumeration type {\tt EnumSort}; sequence type can be encoded as {\tt ArraySort}. The definitions of {\tt Event} and {\tt Trace} are as follows:

\begin{mdframed}[roundcorner=5pt,shadow=true]
\begin{Verbatim}[fontsize=\small]
  Event_lift, (A, V, nil, NDF) = 
    EnumSort('Event_lift', 
            ['A', 'V', 'nil', 'NDF'])

  Trace = ArraySort(IntSort(),Event_lift)
\end{Verbatim}
\end{mdframed}

Because VDM uses partial functions, to encode as SMT formulas in Z3, we need to lift {\tt Event} to include the undefined case {\tt NDF}. Also, square brackets of {\tt [Evnet]} means the value can be {\tt nil}. Furthermore, since {\tt Trace} is defined as an array in Z3, we added constraints that limit an array as a sequence. Note that in Z3, the constraints are applied on each instance of type {\tt Trace} ({\tt tr} in the following formulas), not on the type itself.

\begin{mdframed}[roundcorner=5pt,shadow=true]
\begin{Verbatim}[fontsize=\small]
  tr = Const('tr',Trace)
  [i,j] = Ints('i j')

  ForAll(i, Implies( i<=0, tr[i]==NDF ) )

  ForAll(i,
    Implies(
      And(i>=1,tr[i]!=NDF),
      ForAll(j,
        Implies(And(j>=1,j<=i),tr[j]!=NDF)
      )
    )
  )

  ForAll(i,
    Implies(
      And(i>=1,tr[i]==NDF),
      ForAll(j,
        Implies(j>=i,tr[j]==NDF)
      )
    )
  )
\end{Verbatim}
\end{mdframed}

The first constraint says that the all indexes of {\tt tr} below 1 should be undefined because in VDM-SL, a sequence's index starts from 1. The second constraint says that if an index is defined, i.e., not undefined, all indexes lower than it should be all defined. The third constraint says tat if an index is undefined, all indexes higher than it should be all undefined. From the above three constraints, we limited an array as a sequence that starts from index 1 to its last element. 

% \begin{itemize}
% \item
% A sequence's index is from 1, so indexes below 1 ($<=0$) should be all undefined.
% $\forall i \in \mathcal{Z}, i<=0. ~tr[i] = NDF $
% \item
% If an index is defined in the sequence, all indexes lower then it should be all defined.
% $\forall i \in \mathcal{Z}. ~tr[i] \neq NDF \rightarrow (\forall j \in \mathcal{Z}, 1<=j \land j <= i. ~tr[j] \neq NDF)$
% \item
% If an index is undefined, all indexes higher then it should be all undefined.
% $\forall i \in \mathcal{Z}. ~(i >= 1 \land tr[i] = NDF) \rightarrow (\forall j \in \mathcal{Z}, j >= i. ~tr[j] = NDF)$
% \end{itemize}

% where $NDF$ stands for undefined. The above formulas are added as constraints whenever the instance {\tt tr} of type {\tt Trace} is used.

Since in the VDM-SL model, the length function of a sequence is used, we also need to define the length function in Z3. Here we defined an uninterpreted function {\tt len\_tr} for {\tt tr} the instance of {\tt Trace}.
% $(len\_tr(tr) = 0 \land tr[1] = 0) \lor (len\_tr(tr) > 0 \land tr[len\_tr(tr)] \neq NDF \land tr[len\_tr(tr)+1] = NDF$

\begin{mdframed}[roundcorner=5pt,shadow=true]
\begin{Verbatim}[fontsize=\small]
len_tr=Function('len_tr',Trace,IntSort())

Or(
  And(
    len_tr(tr)==0,
    tr[1]==NDF
  ),
  And(
    len_tr(tr)>0,
    tr[len_tr(tr)]!=NDF,
    tr[len_tr(tr)+1]==NDF
  )
)
\end{Verbatim}
\end{mdframed}

{\tt len\_tr} is defined as an uninterpreted function that takes {\tt tr} as input and returns an integer indicates the length of {\tt tr}. Since the length of a sequence can only be equal or greater than 0, the constraint of {\tt len\_tr} is divided into two cases:

\begin{enumerate}
\item
If {\tt len\_tr} returns 0, then {\tt tr} is an sequence with no element. That is, {\tt tr[1]} has to be undefined so that all indexes of {\tt tr} are then undefined based on previous constraints.
\item
If {\tt len\_tr} returns an integer greater than 0, then {\tt tr} has defined elements till its length index {\tt len\_tr(tr)} such that the element at and after index {\tt len\_tr(tr)+1} should be undefined.
\end{enumerate}

We have defined the context information related to type {\tt Trace}. Now we can procceed to encode and prove the POs of the model. For this VDM-SL model, Overture generated nine POs. Most of the POs are similar so we only demonstrate PO8 and PO3.

$\bullet$ PO8 is a legal sequence application obligation of operation {\tt Pace} which is explicitly specified. This PO requires that the formulas specified with trace {\tt tr} be legal, i.e., computable to get its result.

\begin{mdframed}[roundcorner=5pt]
\begin{Verbatim}[fontsize=\small]
(forall tr:Trace, aperi:nat1, vdel:nat1,
  oldstate & 
    (forall i in set (inds (tl tr)) &
      (((i mod aperi) = (vdel + 1)) =>
        (i in set (inds tr)))
))
\end{Verbatim}
\end{mdframed}

In PO8 showed above, {\tt tl tr} is the tail of {\tt tr}, and {\tt inds} is the operator of gathering indexes of defined elements of {\tt tr} as the set of natural numbers. To prove PO8, firstly we took negation of PO8 which results a quantifier eliminated formula.

\begin{mdframed}[roundcorner=5pt]
\begin{Verbatim}[fontsize=\small]
Exists tr:Trace, aperi:nat1, vdel:nat1 &
  (Exists i in set (inds (tl tr)) &
     Not(
      ((i mod aperi) = (vdel + 1)) =>
       (i in set (inds tr))
     )
  )
\end{Verbatim}
\end{mdframed}

Note that the {\tt oldstate} is not used in the operation so that we remove it in the above formula. To encode the PO in Z3, the PO can be recognized as the formula $\exists~[tr:Trace,aperi:nat1,vdel:nat1,i:nat] P \land Q \land \neg R$ where $P$, $Q$, and $\neg R$ can be encoded separately.

\begin{mdframed}[roundcorner=5pt,shadow=true]
\begin{Verbatim}[fontsize=\small]
  And(aperi>=1,vdel>=1)        #nat1
  And(i>=1, i<=len_tr(tr)-1)   #P
  i%aperi==vdel+1              #Q
  Not(And(i>=1,i<=len_tr(tr))) #not R
\end{Verbatim}
\end{mdframed}

In the encoding of $P$, the tail related formula can be encoded as in the range of $[1,(len~tr)-1]$ so that it is not required to define {\tt tl} and {\tt inds} for sequences. Finally, we checked the satisfiability of the encoded Z3 code and got {\tt unsat}, which means that the PO is proved since we have negated it.


% {\tt Periodic}: legal sequence application obligation\\
% This PO is meant to represent the applicability of statements used in the function {\tt Periodic}, that is, the expressions in {\tt Periodic} are computable to get a result. The formula of the PO is in the form of 
% $\forall tr,e,p. ~\forall t \in inds ~tr. ~(~P \rightarrow ~(~Q \rightarrow ~(~R \rightarrow ~(~S \rightarrow ~(~\forall i. ~T \rightarrow U~)~)~)~)$. where 
% \begin{itemize}
% \item
% $tr: Trace$;~$e:Event$;~$p:nat1$;~$t:nat$
% \item
% $P = t \in set~(inds~tr)$
% \item
% $Q = (tr(t)=e)$
% \item
% $R = (t+p) ~<=~ (len~tr)$
% \item
% $S = (tr((t+p))=e) ~\land~ \forall i \in \{(t+1, \ldots, (t+p)-1\}. ~tr(i) \neq e )$
% \item
% $T = ((t+p) > (len~tr)~)$
% \item
% $U = (\forall i \in \{(t+1), \ldots, len~tr \}. ~i \in set (inds~tr)$
% \end{itemize}


% After taken negation of the formula, we got 
% $\exists tr,e,p,t,i. ~(P \land Q \land Q \land S \land T \land \neg U~)$. In the negated case, the expected result of the negated PO is unsatisfiable, i.e., unsat. The


$\bullet$ PO3 is the postcondition satisfiable obligation of the implicitly specified operation {\tt FaultHeart} that generates a trace of heart pulse signals. PO3 says that the postcondition of {\tt FaultHeart} has to be satisfiable so that the specification of {\tt FaultHeart} is implementable.

% \medskip
\begin{mdframed}[roundcorner=5pt]
\begin{Verbatim}[fontsize=\small]
exists tr:Trace &
  post_FaultyHeart(oldstate, tr, newstate)

FaultyHeart() tr : Trace
post len tr = 100 and
  Periodic(tr,<A>,aperiod) and 
  not Periodic(tr,<V>,aperiod);
\end{Verbatim}
\end{mdframed}
% \medskip

The above VDM-SL code also shows that definition of {\tt FaultyHeart}. Recall that the state is defined as a pair of natural numbers {\tt aperiod} and {\tt vdelay}. The postcondition of {\tt FaultyHeart} is not specified with {\tt oldstate} which has suffixed tilde, i.e, {\tt aperiod\textasciitilde} and {\tt vdelay\textasciitilde}. Therefore, PO3 is a simplified postcondition satisfiable obligation saying that there exists a {\tt Trace} {\tt tr} which satisfies the postcondition specified with only {\tt newstate}\footnote{How the {\tt newstate} is calculated from {\tt oldstate} is not specified in {\tt FaltyHeart}.}. The postcondition uses the function {\tt Periodic} which is explicitly specified and returns a boolean value. We need to encode {\tt Periodic} as well to prove PO3.

\begin{mdframed}[roundcorner=5pt]
\begin{Verbatim}[fontsize=\small]
Periodic: Trace * Event * nat1 -> bool
Periodic(tr,e,p) ==
  forall t in set inds tr &
   (tr(t) = e) =>
   (t + p <= len tr =>
   ((tr(t+p) = e and
     forall i in set {t+1, ..., t+p-1} &
       tr(i) <> e)) and
    (t + p > len tr =>
     forall i in set {t+1, ..., len tr} &
       tr(i) <> e));
\end{Verbatim}
\end{mdframed}

While the specification of {\tt Periodic} is relatively long. The encoding of {\tt Periodic} is similar to the encoding of PO8. We may observe some similar computations on the trace {\tt tr} such as {\tt in set inds tr} and {\tt len tr}, and we can encode them in the same way as handling PO8. Note that we did not apply negation to the formula since existence quantifier in the prefix of a formula is preferred in Z3. As a result, we got a result of {\tt sat} so that PO3 is proved.

The results of all nine POs of the Abstracted Pacemaker VDM-SL model is showed in Table~\ref{tbl:result1} with information of whether negation is applied, and the time used.

\begin{table}[htb]
\begin{center}
\begin{tabular}{|c|c|r|r|}
\hline
PO\#	&	negated	&	result	&	time (sec.) \\ \hline
1		&	Y		&	sat		&	0.031 \\ \hline
2		&	Y		&	unsat	&	0.031 \\ \hline
3		&	N		&	sat		&	15.109 \\ \hline
4		&	Y		&	unsat	&	0.032 \\ \hline
5		&	Y		&	unsat	&	0.046 \\ \hline
6		&	Y		&	unsat	&	0.048 \\ \hline
7		&	Y		&	unsat	&	0.062 \\ \hline
8		&	Y		&	unsat	&	0.031 \\ \hline
9		&	Y		&	sat		&	0.047 \\ \hline
\end{tabular}
\end{center}
\caption{Absract Pacemaker Result}
\label{tbl:result1}
\end{table}


\subsection{Telephone Exchange}

This model specifies an abstracted telephone exchange system. In this model, the operations specify the events which can be initiated either by the system or by a subscriber (user) with implicit style. The system state monitors the calling status of users and the connecting status among users. Type in the model are based on quote types used to indicate the discrete states of users, then maps are specified to relate the calling and connecting status to users as the system state. An invariant is specified for the state {\tt Exchange} to declare the constraints among users and their status in the system.

\begin{mdframed}[roundcorner=5pt]
\begin{Verbatim}[fontsize=\small]
Subscriber = token;
Initiator =  <AI> | <WI> | <SI>;
Recipient = <WR> | <SR>;
Status = <fr> | <un> |
         Initiator | Recipient;
                                                                      
state Exchange of
  status: map Subscriber to Status
  calls:  inmap Subscriber to Subscriber
inv mk_Exchange(status, calls) == 
  forall i in set dom calls & 
    (status(i) = <WI> and
     status(calls(i)) = <WR>) or
    (status(i) = <SI> and
     status(calls(i)) = <SR>)
init s == s = mk_Exchange({|->},{|->})
end
\end{Verbatim}
\end{mdframed}

The strategy of encoding the above types is (1) applying user-defined types in Z3 to encode the quote types; (2) applying uninterpreted functions to encode map type. Here we only demonstrate a few selected types since the others are similar. 

\begin{mdframed}[roundcorner=5pt,shadow=true]
\begin{Verbatim}[fontsize=\small]
Recipient = Datatype('Recipient')
Recipient.declare('WR')
Recipient.declare('SR')
Recipient = Recipient.create()

Status = Datatype('Status')
Status.declare('fr')
Status.declare('un')
Status.declare('INITIATOR',
               ('get_initiator', Initiator))
Status.declare('RECIPIENT',
               ('get_recipient', Recipient))
Status = Status.create()

Status_lift = Datatype('Status_lift')
Status_lift.declare('STATUS',
                    ('get_status', Status))
Status_lift.declare('NDF')
Status_lift = Status_lift.create()

status = Function('status',
                  Subscriber,Status_lift)
calls = Function('calls',
                 Subscriber,Subscriber_lift)
\end{Verbatim}
\end{mdframed}

From above code, the type {\tt Recipient} is defined as an user-defined type with values {\tt WR} and {\tt SR}. Type {\tt Status} is further defined as a user-defined type including {\tt Recipient} and {\tt Initiator}. Note that since VDM uses partial functions, we also defined a lifted type of {\tt Status}. Though {\tt Subscriber} is of token type which does not have specific values when declared, we treated token types as quote types with predefined values such as {\tt S1}, {\tt S2}, and so on. Finally the two maps {\tt status} and {\tt calls} are defined as uninterpreted functions. Overture generated 27 POs for this model. Here we present the encoding and proving of PO1 and PO14.

$\bullet$ PO1 is a legal map application obligation that checks whether maps in the system state are applicable. 

\begin{mdframed}[roundcorner=5pt]
\begin{Verbatim}[fontsize=\small]
forall
  mk_Exchange(status, calls):EXCH`Exchange &
    (forall i in set (dom calls) &
      (i in set (dom status))
    )
)
\end{Verbatim}
\end{mdframed}

Since PO1 explicitly uses the system state, we need to consider the invariant of the system state as the context information of PO1. 
Instead of demonstrate the full encoding of the invariant, we only demonstrate the key encoding for {\tt i in set (dom calls)} and {\tt i in set (dom status)}. Though there is set inclusion used, instead of defining a set, we can encode the set inclusion as whether {\tt calls(i)} is defined or not.

\begin{mdframed}[roundcorner=5pt,shadow=true]
\begin{Verbatim}[fontsize=\small]
  calls(i) != Subscriber_lift.NDF
  status(i)!= Status_lift.NDF
\end{Verbatim}
\end{mdframed}

For PO1, we applied negation to the formula so that the {\tt forall} quantifier for maps {\tt calls} and {\tt status} is removed. The result was {\tt unsat} which means PO1 was proved.

\begin{mdframed}[roundcorner=5pt,shadow=true]
\begin{Verbatim}[fontsize=\small]
Not(
  ForAll(i,
    Implies(
      calls(i)!=Subscriber_lift.NDF,
      status(i)!=Status_lift.NDF
    )
  )
)
\end{Verbatim}
\end{mdframed}

$\bullet$ PO14 is an enumeration map injectivity obligatoin of the operation {\tt Answer} of the system. PO14 has free variables, {\tt r} and {\tt status}, in its formula. These free variables are treated as the quantifier {\tt forall} is prefixed. There are also intermediate variables in the formula such as {\tt m1}, {\tt m2}, {\tt d3}, and {\tt d4}. Note that the inverse function of a map, and the range restriction function {\tt :>} are {\tt applied} on calls and {\tt status}. Also note that {\tt calls} is a one-to-one map {\tt inmap}.

\begin{mdframed}[roundcorner=5pt]
\begin{Verbatim}[fontsize=\small]
(r in set (dom (status :> {<WR>}))) =>
  (forall m1, m2 in 
    set {{r |-> <SR>}, 
         {(inverse calls)(r) |-> <SI>}} &
    (forall d3 in set (dom m1),
            d4 in set (dom m2) &
      ((d3 = d4) => (m1(d3) = m2(d4)))
))
\end{Verbatim}
\end{mdframed}

We negated PO14 so that the {\tt forall} quantifiers are removed. To encode the negated formula of PO14, similar technique for  encoding {\tt set (dom status)} for PO1 can be applied. For the range restriction function, since the range is a singleton set with element value of {\tt <WR>}, we can skip encoding set notations while limiting the map value {\tt status(r)} to {\tt <WR>}. Thus, {\tt r in set (dom (status :> \{<WR>\}))} is encoded as follows.

\begin{mdframed}[roundcorner=5pt,shadow=true]
\begin{Verbatim}[fontsize=\small]
status(r)==
  Status_lift.STATUS(
    Status.RECIPIENT(Recipient.WR) )
\end{Verbatim}
\end{mdframed}

For the map inverse function, we chose not to define a new map as the inverse of {\tt calls}, but encoded {\tt m1 = \{(inverse calls)(r) |-> <SI>\}} as a case of {\tt m1 in set \{\{r |-> <SR>\}, \{(inverse calls)(r) |-> <SI>\}\}}. The case was encoded as the collection of several conditions. 

\begin{mdframed}[roundcorner=5pt,shadow=true]
\begin{Verbatim}[fontsize=\small]
# inverse map is not empty
calls(i)!=Subscriber_lift.NDF 
# inverse_calls(r) == i
Subscriber_lift.get_subscriber(calls(i))==r

m1(i)==Status_lift.STATUS(
         Status.INITIATOR(Initiator.SI)),

ForAll(k, 
  Implies(k!=i, m1(k)==Status_lift.NDF) )
\end{Verbatim}
\end{mdframed}

In above code, the first two conditions promise that (1) the part of map, where key {\tt i} is used and inverse is applied, has to be defined; (2) the domain of the inversed map, i.e., the range of the original map, is {\tt r}. Then {\tt m1(i)} where key {\tt i} is referred should be {\tt <SI>}. Finally the last condition guarantees that {\tt m1} should be undefined for all indexes except {\tt i}.

Recall that {\tt calls} is defined as {\tt inmap}, a one-to-one map. Thus, we had to add  the conditions about {\tt inmap} as context information: for any two different users, either both users calls no one (undefined) or they are calling different users.

\begin{mdframed}[roundcorner=5pt,shadow=true]
\begin{Verbatim}[fontsize=\small]
k  = Const('k',Subscriber)
l  = Const('l',Subscriber)
ForAll([k,l],
  If(
    k==l,
    calls(k)==calls(l),
    Or(
      And(
        calls(k)==Subscriber_lift.NDF,
        calls(l)==Subscriber_lift.NDF
      ),
      calls(k)!=calls(l)
    )
  )
)
\end{Verbatim}
\end{mdframed}
\medskip

The result of checking PO14 (negated) ware expected to be {\tt unsat}, but we got {\tt sat}. Z3 returned a model that satisfies the negated formula and constraints so that we could examine where is the problem. We found that {\tt calls} in the returned model is a map where users calls themselves. This does not violate the type of {\tt inmap}, but an user cannot call him/her-self in a telephone exchange system. Therefore, the checking of PO14 suggested that the VDM-SL model should add a constraint on {\tt calls}. As an example of correction, we may add the constraint below as the type invariant of {\tt calls} or the additional state invariant\footnote{We have added the constraint and checked again, and the PO was proved successfully.}.

\begin{mdframed}[roundcorner=5pt]
\begin{Verbatim}[fontsize=\small]
forall i in set dom calls & calls(i) != i
\end{Verbatim}
\end{mdframed}

In this case study, we did not check all the 27 POs since the encoding could not cover all POs. The main issue is that the maps are encoded as uninterpreted functions so that {\tt forall} quantifier cannot be applied. This results that we could not check the postcondition satisfiable obligations of this model\footnote{Besides postcondition satisfiable obligations, there are two POs with type check function {\tt is\_(name, type)}, for which Z3 encoding is not available and then skipped.}. Finally, we checked 16 out of all 27 POs.

\subsection{Discussion}

% Overture generated 27 POs for this model. These POs can be divided into several groups as showed below. In this case study we encoded and checked the first three groups using the above encoding of types. The reason of not handling the postcondition obligations is that the maps defined using uninterpreted functions cannot be used together with quantifiers. Also, map inverse obligations have {\tt is\_(name, type)}, the function of type check provided by VDM, in their formulas. There is no appropriate encoding for such type check functions in Z3. 

% \begin{enumerate}
% \item
% Legal map application obligations, invariant satisfiable obligations: 1 to 7.
% \item
% Enumeration map injectivity obligations: 9, 14, 18, 21, 25.
% \item
% Legal map application obligations with inverse map: 12, 17, 20, 23.
% \item
% Operation postcondition satisfiable obligations: 8, 10, 11, 15, 16, 19, 22, 26, 27
% \item
% Map inverse obligations: 13, 24.
% \end{enumerate}

Points of discussion:
(1) Automation: Is the encoding can be automated? partly or fully? 
(2) Guidance and patterns: What patterns are repeated used in encoding? Any guide of encoding?



\section{Related Work}
\label{section:related-work}

% Development history of VDM POG.
Proof is essential to guarantee the consistency of VDM models~\cite{978-3-540-19813-0}, and the proof obligation generation of VDM is firstly proposed by B. K. Aichernig and P. G. Larsen~\cite{AL:97:POGV}. Later, A. Ribeiro and P. G. Larsen worked on proof obligation generation and discharging related to the termination of recursive functions~\cite{Ribeiro2010}.

% Work related to discharging POs using theorem provers. 
There are researches focusing on translation from VDM to theorem provers and discharging POs of VDM with theorem provers. S. Agerholm~\cite{Agerholm1996} proposed a translation from a subset of VDM-SL to PVS for type checking. S. Maharaj and J. Bicarregui~\cite{632849} used Agerholm's translation to assist the verification of VDM-SL models and their refinements. Later, S. D. Vermolen et al.~\cite{Verm:2007:master,Vermolen:2010:PCV:1774088.1774608} utilized the parsing mechanism of Overture tools and developed automated translation from a subset of VDM-SL to HOL theorem prover. Vermolen's work also provided some tactics to support automated discharging of POs. L. D. Couto et al.~\cite{CFP:14:TVCSAP} used the proof obligation generation mechanism of Overture tools for translating CML, a modeling language combing VDM-SL and CSP, to Isabella/HOL for automated proof. Recently, work on improvement of translating VDM-SL to Isabella/HOL was proposed~\cite{CT:15:EOCGTIS}.

% Comparison to our approach
The theorem prover based approaches has common issues: (1) the translation from VDM-SL to theorem provers may be erroneous since the translation applies to the whole VDM-SL model; (2) The proof process is usually complicated and requires an expert even with the help of proof tactics; (3) Theorem provers provide little information when they fail to discharge POs. Compare to our approach that discharges POs of VDM-SL with SMT solvers, we adopt the efficiency of SMT solvers and the encoding does not require to be applied on the whole VDM-SL model. Furthermore, the counterexample generation of SMT solvers is helpful for locating the problem. However, encoding in our approach is manual at this point and still need improvements.

% Work related to validation/verfication of VDM models with model checking.
On the other hand, verification and validation of VDM-SL models with model checking techniques are recently proposed. K. Lausdahl~\cite{kenneth:ifm2013} proposed a semantics-preserving translation that constructs an Alloy model from a subset of VDM-SL. This work aims to support the validation of implicitly specified VDM-SL models with Alloy. H-H. Lin et al.~\cite{DBLP:conf/ftscs/LinOKA15} proposed an approach that combines SPIN model checker and the VDMJ interpreter for verifying explicitly specified VDM-SL models. Lausdahl's work is similar to our approach since Alloy uses SAT solvers, but SMT solvers in our approach can solve wider formulas with more efficiency. Both Lausdahl's and Lin's work require types involved in a VDM-SL model being bounded, and there are issues for Alloy and SPIN to represent the rich and complex VDM-SL types. Compare to them, bounds are not necessary for SMT solvers in our approach, and our encoding patterns can be applied to complex types thanks for built-in theories in SMT solvers.

% For example, \cite{Verm:2007:master,Vermolen:2010:PCV:1774088.1774608} translate VDM models to HOL and then prove the POs within the environment of HOL. However, this approach has several issues: (1) the translation is applied to the whole VDM model so that it may be tedious and error-prone; (2) The proof process is usually complicated and requires an expert; (3) It is not clear where the problem is when the proof fails.

% \cite{CFP:14:TVCSAP,CT:15:EOCGTIS}

\section{Conclusions}
\label{section:conclusions}




% use section* for acknowledgment
%\section*{Acknowledgment}


%The authors would like to thank...


% trigger a \newpage just before the given reference
% number - used to balance the columns on the last page
% adjust value as needed - may need to be readjusted if
% the document is modified later
%\IEEEtriggeratref{8}
% The "triggered" command can be changed if desired:
%\IEEEtriggercmd{\enlargethispage{-5in}}

% references section

% can use a bibliography generated by BibTeX as a .bbl file
% BibTeX documentation can be easily obtained at:
% http://mirror.ctan.org/biblio/bibtex/contrib/doc/
% The IEEEtran BibTeX style support page is at:
% http://www.michaelshell.org/tex/ieeetran/bibtex/
\bibliographystyle{IEEEtran}
% argument is your BibTeX string definitions and bibliography database(s)
\bibliography{IEEEabrv,refs}
%
% <OR> manually copy in the resultant .bbl file
% set second argument of \begin to the number of references
% (used to reserve space for the reference number labels box)
%\begin{thebibliography}{1}

%\bibitem{IEEEhowto:kopka}
%H.~Kopka and P.~W. Daly, \emph{A Guide to \LaTeX}, 3rd~ed.\hskip 1em plus
%  0.5em minus 0.4em\relax Harlow, England: Addison-Wesley, 1999.

%\end{thebibliography}




% that's all folks
\end{document}


