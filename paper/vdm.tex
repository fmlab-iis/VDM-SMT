% Give a simple example with the syntax used in case studies and explain implicit and explicit specification in VDM-SL.

% General information of VDM
The Vienna Development Method (VDM) was originally developed in the 1970's at the IBM laboratories in Vienna~\cite{DBLP:conf/fm/1978}. The VDM Specification Language (VDM-SL) is a higher-order language with formally defined syntax and semantics~\cite{Prehn:1991:LNCS551,Larsen1995585}. VDM provides various abstract data types: basic types such as booleans, natural numbers, and tokens; advanced types such as record, product, set, and map. Typed variables (state) may be restricted by invariants and operations/functions may be specified with preconditions and postconditions.

% Explaining the sample VDM-SL model: part1
As a brief explanation of VDM-SL, Fig.~\ref{fig:module_sample} shows a sample VDM-SL model {\tt CMDS}. The specification is divided into several blocks: types (line 1-4), state (line 6-12), operations (line 14-18), and functions (line 20-27). {\tt CMD} is a union type of quotes which is usually used for enumeration. {\tt CMD} represents commands {\tt <R>} and {\tt <L>} which may be recognized as right and left. {\tt CMD\_series} is defined as a sequence type which is a list of commands. Note that the type of elements {\tt [CMD]} with square brackets means the union type {\tt POS|nil}. We also defined a map type for counting the occurrences of commands. The state of the model is defined as a sequence of commands {\tt CMD\_series} and its initial value is defined as empty. Its invariant specifies that adjacent elements of {\tt commands} have to be different. The invariant is specified with {\tt forall} quantifier and {\tt len} operation that returns the length of a sequence.


% An example of VDM-SL model (or some segments of VDM-SL code) should have
% types: quote and union, sequence, token, map
% operations of sequence: len, hd, tl
% operations of map: range restriction
% operation of set: in set
% others: state def., forall statement
\begin{figure}[t]
\begin{center}
\begin{mdframed}[roundcorner=5pt]
\begin{Verbatim}[fontsize=\small,numbers=left]
types
CMD = <R> | <L>;
CMD_series = seq of [CMD];
CMD_times = map CMD to nat;

state S of
  commands : CMD_series
  inv s == forall k in set 
    {1,...,len s.commands - 1} &
    s.commands(k) <> s.commands(k+1)
  init p == p = mk_S([])
end

operations
  push_cmd(a:[CMD])
  pre commands = [] or hd commands <> a
  post hd commands = a and
       tl commands = commands~;

functions
  times_count : CMD_series -> CMD_times
  times_count(a) == {
    <R> |->
    len [ i | i in set inds a & a(i)=<R> ],
    <L> |->
    len [ i | i in set inds a & a(i)=<L> ]
  };
\end{Verbatim}
\end{mdframed}
\caption{A sample VDM-SL specification: CMDS}
\label{fig:module_sample}
\vspace{-25pt}
\end{center}
\end{figure}

% Explaining the sample VDM-SL model: part2
The operation {\tt push\_cmd} is specified to accept a command {\tt CMD} or {\tt nil} and pushes the command as the head of {\tt commands}. The postcondition is specified using the old state {\tt commands\textasciitilde} with postfixed tilde and says that the old state is the tail of the new state {\tt commands}. Note that {\tt push\_cmd} is an implicit operation that only specifies the signature with pre- and post-conditions. The function that counts the occurrences of commands is specified explicitly. Instead of using loop, the function uses sequence comprehension to construct the map of type {\tt CMD\_times}. This also shows the expression power of VDM in specification. 

% tools and dialects of VDM (optional)
Existing tools such as Overture Tool~\cite{Larsen:2010:OII:1668862.1668864} and VDMTools~\cite{2008:VAS:1361213.1361214} provide graphical user interfaces for easy editing/building of VDM models. These tools also provide functionalities such as type/syntax checking and testing/animation (execution by the interpreter~\cite{Prehn:1991:LNCS551}) for validation and verification of VDM models. There are other dialects of VDM: VDM++ and VDM Real-Time (VDM-RT)~\cite{10.1007/11813040_11}. VDM++ is the extension of VDM-SL with object-orient concepts; VDM-RT further extends VDM++ with scheduling controls of threads or processes.
