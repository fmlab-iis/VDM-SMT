% Development history of VDM POG.
% Proofs are essential to guarantee the consistency of VDM models~\cite{978-3-540-19813-0},
Proof obligation generation of VDM is firstly proposed by B. K. Aichernig and P. G. Larsen~\cite{AL:97:POGV}. Later, A. Ribeiro and P. G. Larsen worked on proof obligation generation and discharging related to the termination of recursive functions~\cite{Ribeiro2010}.
% Work related to discharging POs using theorem provers. 
The mainstream of discharging POs of VDM models rely on translation from VDM to theorem provers. The generation and discharging of POs are conducted on theorem provers such as PVS and Isabella/HOL. S. Agerholm~\cite{Agerholm1996} proposed a translation from a subset of VDM-SL to PVS for type checking. S. Maharaj and J. Bicarregui~\cite{632849} used Agerholm's translation to assist the verification of VDM-SL models and their refinements. Later, S. D. Vermolen et al.~\cite{Verm:2007:master,Vermolen:2010:PCV:1774088.1774608} utilized the parsing mechanism of the Overture tool and developed automated translation from a subset of VDM-SL to HOL theorem prover. 
% Vermolen's work also provided some tactics to support automated discharging of POs. L. D. Couto et al.~\cite{CFP:14:TVCSAP} used the proof obligation generation mechanism of the Overture tool for translating CML, a modeling language combing VDM-SL and CSP, to Isabella/HOL for automated proof. 
Recently, work on improvement of translating VDM-SL to Isabella/HOL was proposed~\cite{CT:15:EOCGTIS}.

% Comparison to our approach
In general, the theorem prover approach has two major issues: 
% (1) the translation from VDM-SL to theorem provers may be erroneous since the translation applies to the whole VDM-SL model; 
(1) The proof process is usually complicated and requires an expert even with the help of proof tactics; (2) Theorem provers provide little information when they fail to discharge POs. Compared to our approach, we adopt the efficiency of SMT solvers, and the counterexample generation of SMT solvers is helpful for locating the problem. However, encoding in our approach is manual at this point and still need improvements.

Theorem proving with SMT solvers is not a new idea. S. Merz and H. Vanzetto~\cite{Merz2012TLASMT} has proposed a backend using SMT solvers to assist discharging proof obligations of TLA{\thinspace}+{\thinspace}. D. Kr\"oning et al.~\cite{smtLSM2009} has proposed a theory of finit sets, lists, and maps for SMT-LIB. The proposed theory is general but limited in basic types of {\tt Int} and {\tt Real}. The authors only discussed how to realize the theory by reduction to existing SMT theories with a report of initial benchmarks\footnote{http://www.cprover.org/SMT-LIB-LSM/} on a Event-B case study. The idea of adopting existing theories of SMT is the same as our approach. However, we argue that a strategic encoding like our approach is necessary for discharging VDM POs than introducing general theories because the infinite and undecidable nature of sets, lists, and maps in VDM.

% Work related to validation/verfication of VDM models with model checking.
% Verification and validation of VDM-SL models with model checking techniques are recently proposed. K. Lausdahl~ \cite{kenneth:ifm2013} proposed a semantics-preserving translation that constructs an Alloy model from a subset of VDM-SL. This work aims to support the validation of implicitly specified VDM-SL models with Alloy. H-H. Lin et al.~\cite{DBLP:conf/ftscs/LinOKA15} proposed an approach that combines SPIN model checker and the VDMJ interpreter for verifying explicitly specified VDM-SL models. Lausdahl's work is similar to our approach since Alloy uses SAT solvers, but SMT solvers in our approach can solve wider formulas with more efficiency. Both Lausdahl's and Lin's work require types involved in a VDM-SL model being bounded, and there are issues for Alloy and SPIN to represent the rich and complex VDM-SL types. Compare to them, bounds are not necessary for SMT solvers in our approach, and our encoding patterns can be applied to complex types thanks for built-in theories in SMT solvers.
