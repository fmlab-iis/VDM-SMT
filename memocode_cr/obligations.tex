% Objective: explain the semantics of Overture generated POs.
% Explain what is proof obligation in VDM. There are kinds of POs: domain, subtype, satisfiability, and termination. Any example?
% Explain POs generated by Overture tool. Any example?

% Generation information of POs
For a VDM model, a proof obligation (PO) is a statement that must be proved to ensure the consistency of the model. A PO contains a predicate with its context information as shown below. The context information is from code segments of the VDM model that relate to the predicate to be proved.

\begin{mdframed}[roundcorner=5pt]
\begin{Verbatim}[fontsize=\small]
PO: context information ==> predicate
\end{Verbatim}
\end{mdframed}

% Classifications of POs and sample POs from sample VDM-SL model
POs of VDM-SL can be classified type compatibility (subtype checking), domain checking, and satisfiability~\cite{AL:97:POGV,Vermolen:2010:PCV:1774088.1774608}. Type compatibility relates to types with invariant or subtypes; domain checking relates to the use of partial functions and partial operators; satisfiability relates postconditions of functions and operations. Both Overture and VDMTools generate POs with the syntax of VDM.

For the sample VDM-SL model shown in Fig.~\ref{fig:module_sample}, Overture generated ten POs. Here selected POs are demonstrated and explained. PO1 is called ``legal sequence application obligation'' that relates to the sequence involved expression (line 9-10) of the VDM-SL model {\tt CMDS}). PO1 says that for all k in the set from 1 to {\tt len s.commands -1}, k should always be within the bound of defined indices of the sequence {\tt s.commands}. Also, it can be recognized that if we use {\tt len s.commands} instead of {\tt len s.commands -1} the proof will fail since a sequence in VDM is indexed from 1 to its length. Also note that in PO1, the state {\tt CMDS'S} is mentioned, and we have to include the state invariant as well though it is not explicitly described in PO1.

\begin{mdframed}[roundcorner=5pt]
\begin{Verbatim}[fontsize=\small]
(forall s:CMDS`S & 
  (forall k in set 
    {1, ... ,((len (s.commands)) - 1)} & 
    (k in set (inds (s.commands)))))
\end{Verbatim}
\end{mdframed}

PO7 is an operation postcondition satisfiable obligation for {\tt push\_cmd} (line 15-18). PO7 says that for every input {\tt a:CMD} and {\tt oldstate}, if the precondition which takes inputs {\tt a} and {\tt oldstate} is satisfied, the postcondition which takes inputs {\tt a}, {\tt oldstate}, and {\tt newstate} should be satisfied. Note that instead of the specification of precondition, postcondition, and state in {\tt CMDS} module, special names {\tt oldstate}, {\tt newstate}, and {\tt post\_push\_cmd} are used. In VDM-SL, the {\tt pre\_} and {\tt post\_} prefix represent the precondition and postcondition of a function/operator separately. They are evaluable functions if given appropriate arguments, and return a boolean value. In a satisfiable obligation of an operation in which the state change is usually expected, the name of state (in this case {\tt commands}) represents {\tt newstate} while the same name postfixed with ``~\textasciitilde~'' (in this case {\tt commands\textasciitilde}) represents {\tt oldstate}.

\begin{mdframed}[roundcorner=5pt]
\begin{Verbatim}[fontsize=\small]
(forall a:[CMD], oldstate:CMDS`S & 
  (pre_push_cmd(a, oldstate) => 
    (exists newstate:CMDS`S & 
      post_push_cmd(a, oldstate, newstate))
))
\end{Verbatim}
\end{mdframed}
