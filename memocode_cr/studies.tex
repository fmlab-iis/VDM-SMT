We have conduct two case studies\footnote{The VDM-SL models are from the Overture repository: {\tt http://overturetool.org/download/examples/VDMSL/}. The encoding of POs done by hand and their checking results can be founded in our repository: {\tt https://github.com/fmlab-iis/VDM-SMT}}: an abstract pacemaker and a telephone exchange system.

% \subsection{Abstract Pacemaker}
% \label{section:case1}

\noindent{\bf Abstract Pacemaker.} ~~The first case is a small VDM-SL model specifies an abstract pacemaker. The model is not a full specification and focuses on the part of definitions and operations related to heartbeat signals detected and recorded in a pacemaker. For this model of 58 lines, the Overture tool generated nine POs. The encoding and checking of POs follow the encoding steps described in Section~\ref{section:encode-strategy}. In this section, we demonstrate some of the encoding patterns for encoding lists with arrays.

As the code shown below, in the abstract pacemaker model, the core types are {\tt Trace} and {\tt Event} since they show in almost all POs of the model. {\tt Trace} is of type sequence of {\tt Event}, where {\tt Event} is of quote type with two values {\tt <A>} and {\tt <V>} that stand for the A-pace and V-pace of heartbeat signals. 

\medskip
\begin{mdframed}[roundcorner=5pt]
\begin{Verbatim}[fontsize=\small]
  Trace = seq of [Event];
  Event = <A> | <V>;
\end{Verbatim}
\end{mdframed}
\medskip

The code below shows the encoding of {\tt Event} and {\tt Trace}. Since {\tt Trace} is not exactly the sequence of {\tt Event} but {\tt [Event]} (a union with {\tt nil}), we did not define {\tt Event} but define {\tt [Event]} directly. Furthermore, a type in VDM can be undefined because of VDM's logic of partial functions. This means that besides predefined values as in the specification, we have to consider the case that the value is undefined. In our work, we used {\tt NDF} to represent the undefined value of {\tt Event}, and defined {\tt Event\_lift} in Z3. The postfixed {\tt \_lift} indicates that we have considered the undefined value.

\medskip
\begin{mdframed}[roundcorner=5pt,shadow=true]
\begin{Verbatim}[fontsize=\small]
  Event_lift, (A, V, nil, NDF) = 
    EnumSort('Event_lift', 
            ['A', 'V', 'nil', 'NDF'])

  Trace = ArraySort(IntSort(),Event_lift)
\end{Verbatim}
\end{mdframed}
\medskip

In the above encoding, {\tt Trace} is encoded as an array type in Z3 and this array type is by default indexed through the integer domain. An array type can not be a sequence type unless we add constraints regarding sequences to limit an array to a sequence. Below shows the encoding of the constraint to {\tt tr}, the instance of {\tt Trace} as an array. Note that in Z3, we cannot add constraints to types but instances. The code describes three constraints to reduce an array ({\tt tr}) to a sequence that starts from index 1 to its last element.

\medskip
\begin{mdframed}[roundcorner=5pt,shadow=true]
\begin{Verbatim}[fontsize=\small,numbers=left]
  tr = Const('tr',Trace)
  [i,j] = Ints('i j')

  ForAll(i, Implies( i<=0, tr[i]==NDF ) )

  ForAll(i,
    Implies(
      And(i>=1,tr[i]!=NDF),
      ForAll(j,
        Implies(And(j>=1,j<=i),tr[j]!=NDF)
      )
    )
  )

  ForAll(i,
    Implies(
      And(i>=1,tr[i]==NDF),
      ForAll(j,
        Implies(j>=i,tr[j]==NDF)
      )
    )
  )
\end{Verbatim}
\end{mdframed}
\medskip


\begin{itemize}
\item
The first constraint (line 4) says that all indices of {\tt tr} less than 1 should be undefined because a sequence's index starts from 1 in VDM.
\item
The second constraint (lines 6-13) says that if an index is defined, i.e., not undefined, all indices lower than it should be defined.
\item
The third constraint (line 15-22) says tat if an index is undefined, all indices greater than it should be undefined.
\end{itemize}

With another encoding for {\tt len} operator of sequences. We were able to prove all nine POs of the abstracted pacemaker model. We also applied some other encoding patterns to encode operators {\tt hd}, {\tt tl}, {\tt in set inds}. These operators can be encoded as constraints on the range of a sequence so that we do not have to declare corresponding instances of sequences or sets.

\noindent{\bf Telephone Exchange.} ~~The second case study is a VDM-SL model specifies an abstracted telephone exchange system. In this model, the operations specify the events which can be initiated either by the system or by a subscriber (user) with implicit style. The system state monitors the calling status of users and the connecting status among users. The Overture tool generated 27 POs, and we managed to encode and prove 16 POs. The main reason of leaving some POs unproved is because of nested universal quantifiers.

Instead of using arrays, we tried using uninterpreted functions to encode maps. We also applied encoding patterns mentioned in the abstract pacemaker case. In the checking, we found an inconsistency about the constraint regarding {\tt calls}, the map specifying relations between subscribers. The counterexamples generated by Z3 were helpful for suggesting how to correct the model.

Table~\ref{tbl:results} shows the results of the two case studies. The checks are conducted on a machine with Intel Core i7 2.4GHz CPU and 8GB RAM.

\begin{table}[htb]
\begin{center}
\begin{tabular}{|c|c|r|r||c|c|r|r|}
\hline
\multicolumn{4}{|l||}{Abstract Pacemaker} & \multicolumn{4}{|l|}{Telephone Exchange} \\
\hline \hline
PO\#	&	neg.	&	res.	&	time(s)	&	PO\#	&	neg.	&	res.	&	time(s) \\ \hline
1		&	+		&	sat		&	0.031	&	1		&	+		&	unsat	&	0.032 \\ \hline
2		&	+		&	unsat	&	0.031	&	2		&	+		&	unsat	&	0.031 \\ \hline
3		&	-		&	sat		&	15.109	&	3		&	+		&	unsat	&	0.031 \\ \hline
4		&	+		&	unsat	&	0.032	&	4		&	+		&	unsat	&	0.032 \\ \hline
5		&	+		&	unsat	&	0.046	&	5		&	+		&	unsat	&	0.03 \\ \hline
6		&	+		&	unsat	&	0.048	&	6		&	+		&	unsat	&	0.047 \\ \hline
7		&	+		&	unsat	&	0.062	&	7		&	-		&	sat		&	0.031 \\ \hline
8		&	+		&	unsat	&	0.031	&	9		&	+		&	unsat	&	0.047 \\ \hline
9		&	+		&	sat		&	0.047	&	12		&	+		&	sat		&	0.015 \\ \hline
		&			&			&			&	14		&	+		&	sat		&	0.063 \\ \hline
		&			&			&			&	17		&	+				&	sat		&	0.016 \\ \hline
		&			&			&			&	18		&	+		&	unsat	&	0.047 \\ \hline
		&			&			&			&	20		&	+		&	sat		&	0.016 \\ \hline
		&			&			&			&	21		&	+		&	unsat	&	0.046 \\ \hline
		&			&			&			&	23		&	+		&	sat		&	0.015 \\ \hline
		&			&			&			&	25		&	+		&	sat		&	0.063 \\ \hline \hline
\multicolumn{4}{|l||}{POs checked: 9 / 9 (100\%)} & \multicolumn{4}{|l|}{POs checked: 16 / 27 (59\%)} \\ \hline		
\end{tabular}
\end{center}
\caption{Results of Case Studies}
\label{tbl:results}
\vspace{-20pt}
\end{table}
% \vspace{-20pt}

